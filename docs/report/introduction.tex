\chapter{Introduction\label{chap:introduction}}
  More people are online than ever before. Comment threads and discussion forums allow us to spend spare time participating as contributors - rather than just consuming content. From the latest blockbuster title to yesterday's celebrity misdemeanor, there's an online conversation already well underway. These discussions, where statements are encoded in natural language, represent a large untapped resource of ideas. A higher-level view of this information would be useful and interesting to many.

  This project explores an approach for summarizing such information. At the core of the approach is the notion of a `point' - a short refined argument statement extracted as a verb and it's context. We model a user's argument in a discussion as consisting of one or more points and use them as content units in summarizing the discussion. Points are first extracted from text and clustered to give a summary of the discussion. To test our approach, we ran an evaluation using summaries generated from online political discussions \cite{walker2012corpus}.

  \section{Motivation}
    Summarization is a well established task in the field of Natural Language Processing. Summarization sub-tasks such as extraction and compression, where text is selected and removed to arrive at a summary, have become commonplace due to the complexities introduced by abstractive methods. Extractive methods have become largely statistical using Naive-Bayes \cite{kupiec1995trainable} approaches and, more recently, even Neural Networks \cite{svore2007enhancing}.

    Argumentation Mining, a newer area of study, has the aim of identifying argumentative discourse structure in text. Argumentation Mining has been successfully used in the processing of formal documents (where arguments are often stated more explicitly) such as parliamentary records \cite{palau2009argumentation} and legal documents \cite{montemagni2010semantic}. However, Argumentation Mining has also more recently been applied to more informal texts \cite{park2015conditional}. Such applications of Argument Mining on informal text, coupled with summarization, encapsulates much of the novelty for this project.

    Using points as a data structure was adopted from a previous project in the department that used a similar concept in a system with a focus on stance classification. While this implemented point identification, the tool was incapable of extracting points beyond subject, verb, object tuples and still used sentence extracts for presentation. The tool was capable of linking contrasting points, but only based on the presence of negation terms.

    Our project was based on this broad concept of a point as well as the idea points, representing units of informal argumentation, could be used to build a high-level summary of a discussion. News article comment sections; forum threads; film \& product reviews and even extended email conversations are all candidate applications for such analysis.

  \section{Objectives}
    Continuing the work of a past project that implemented the idea of a simple point, we wanted to build on this and improve on the extraction of points from text. Fundamentally, this meant extracting more of the verb's arguments and related tokens - rather than only the subject and object. Take the following sentence:

    \medskip
    \begin{center}
    \blockquote{\textit{I don't think so, an unborn child (however old) is not yet a human being.}}
    \end{center}
    \medskip

    From this, taking only the subject and object of the verb \textit{is}, gives: \texttt{child.subject be.verb human.object}. This is useful, a structure like this can be linked with a point extracted from a sentence like: \blockquote{\textit{So you say: children are not complete humans until birth?}}. However, this lossy attraction is not suited to presentation. Our objective, with regards to better extracting points, was to \textit{additionally} extract the sub-sentence context in which a point was made. Continuing this example, the following string would be a `good' extract:

    \medskip
    \begin{center}
    \blockquote{\textit{an unborn child is not yet a human being}}
    \end{center}
    \medskip

    This almost half the length and still expresses the core idea expressed around `is' in the above sentence. Another goal was to go beyond just extracting points and investigate relationships between them. Counter points pairs --- points that represent opposing ideas --- and co-occurring points - points commonly raised by the same user - were of particular interest. Connecting points and a user's annotated stance was also discussed. We also set out to present this information in a way that was easy to interpret.

    \medskip

    The project's success is evaluated with respect to these objectives.
