\chapter{Introduction\label{chap:introduction}}
  More people are online than ever before. Comment threads and forums allow us to spend our spare time participating as contributors - rather than being just content consumers. From the latest blockbuster title to yesterday's celebrity misdemeanor there's an online conversation already well underway.

  These discussions, where statements are encoded in natural language, represent a large untapped resource of ideas. A higher level view of this ever-evolving data set would be of interest to many working the social sciences as well as the participants of online discussions.

  This project explores an approach for summarization of such information. At the core of the approach is the notion of a `point' - a short refined argument statement. We model a user's argument in a discussion as consisting of one or more points. We use these points as the content units when summarizing the discussion. Points are extracted from text and grouped to give a summary of the discussion. We then test our implementation's performance by running an evaluation using summaries generated from various political debates sourced from online discussions \cite{walker2012corpus}.

  \section{Motivation}
    Summarization has been a long-running task in the field of Natural Language Processing. Summarization sub-tasks such as extraction and compression, where text is selected and removed to arrive at a summary, have become commonplace due to the complexities introduced by abstractive methods. Extractive methods have become largely statistical using Naive-Bayes \cite{kupiec1995trainable} approaches and, more recently, Neural Networks \cite{svore2007enhancing}.

    Argumentation Mining, a newer area of study, has the aim of identifying argumentative discourse structure in text. Argumentation Mining has been successfully used in the processing of formal texts such as parliamentary records \cite{palau2009argumentation} and legal documents \cite{montemagni2010semantic} where arguments are often stated more explicitly. However, Argumentation Mining has also more recently been applied to more informal texts \cite{park2015conditional}. Such applications of argument mining on informal texts, coupled with summarization, encapsulates much of the novelty for this project.

    The idea behind using points was adopted from a previous project from the department that used a similar concept in a system for stance classification. This had point identification, however, the tool was not capable of extracting units of text smaller than sentences. The tool was capable of linking contrasting points, but only based on the presence of negation terms.

    The project was based on the concept of a point as well as the idea that informal argumentative discourse could be used to build a high-level summary of an online discussion. News article comment sections; forum threads; film \& product reviews and even extended email conversations are all candidate applications for such a tool.

  \section{Objectives}
    Starting with the definition for a point: a verb and it's dependents, we set the following objectives for the project.

    Leading on from the stance classification project \textbf{(cite Angrosh?)}, we wanted to improve on the extraction of points from text. This meant ensuring a complete list of a verb's dependents was required for the presentation of points - rather than using the (often verbose) containing sentence.

    Another goal was to investigate relationships between points such as contrastive or co-occurring points - this involved expanding the ways in which related points could be matched, beyond negation. As secondary objective, we wanted to investigate supporting points. These were points that not only commonly co-occured in posts but also had a place in an argument structure.

    The task of stance classification was also discussed. Another secondary objective was to investigate if certain points were representative for a given known stance. Stance was annotated on some of the corpus debates.
    \textbf{(do things that we didn't manage to get to become comments for further work instead?)}

    Finally we set out to present this information in in a way that was easy to interpret. This evolved into our debate summaries.
