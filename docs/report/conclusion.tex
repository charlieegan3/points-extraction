\chapter{Summary \& Conclusion \label{chap:conclusion}}
  \section{Summary}
    In this project we have implemented a robust method for extracting points, a meaningfully shorter content unit than sentences. We made use of these in a summarization task by clustering points into cohesive sections. We then evaluated the effectiveness of our approach by comparing our summaries against those generated by a baseline statistical tool that uses sentence extraction.

    We were able to meet most of the project's initial goals. Using dependency parses, we have implemented a means of extracting more useful and complete points than those in previous project. We also improved on the detection of counter points by expanding the approach to account for antonyms --- rather than using negation terminology alone. While we also implemented an approach for the identification of co-occurring points, the results seem to be more variable and are highly dependent on large discussions where participants make many points.

    We opted to present the results of the analysis as discussion summaries. Here we were able to go beyond our goals of counter/co-occurring points and include additional sections based on the points we had extracted. There are however areas that require further work. Results from our evaluation suggest that certain presentation decisions were not always popular. We also found that while scores allocated my our bigram model correlated with those of participants, this approach in selecting extracts from clusters was not significantly better than random selection (from the same clusters) when a compared at the summary level.

    The results of the summary comparisons were very positive with all our summary types performing significantly better in comparisons. We were able to reject our first two hypotheses. We were not able to reject \textbf{H3} but have gathered useful information for future work regarding extract selection. We were unable to gather enough targeted feedback regarding sections to reject \textbf{H4} but again gathered much useful information for improving on the tool.

    Comments from study 3, at the social media workshop, suggested that the summaries fell between quantitative and qualitative analysis and this they could be made more useful to researchers by quoting the (already available) ratio of counter-point sides. Comments also suggested it would be useful to select the required summary length as well as provide links back to the discussion source text. The disagreement section was referenced as being the most useful while participants found related points confusing.

  \section{Further Work}
    The implementation is still the product of an experimental development process. Some components of the summary generation module have poor maintainability. Next steps include reducing the agglomerative complexity and repetition as well as establishing a level of unit testing. Clearer definitions for the re-formatting of extracts and extract selection (likely best expressed as a DSL) would also be beneficial for code readability. This ties into the corpus pre-processing and extract presentation, both areas that require a more integrated implementation.

    The tool chain is currently closely coupled with the corpus. We would like to establish a less bespoke approach to make analysis of new corpora easier. One interesting direction would be a simple web application that capable of generating a summary for a discussion of the user's choosing from \textit{Reddit}, \textit{Hacker News} or online forum. This potential use case raises issue of the required discussion length. Currently a long discussion (upwards of 100,000 words) is required to extract a sufficient number of points to fill all our summary sections without repetition --- and build a good sample of counter/co-occurring points. This would likely require the summary structure to be more flexible, generating shorter summaries for shorter discussions and contextually removing sections where information was lacking.

    From the evaluation results we found that our bigram extract selection approach was not providing reliably better results than random --- though it's scores correlated with those of participants. This appeared to come down to both readability and informativeness. Currently extracts are weighted by the length, this is perhaps dominating the bigram score intended to represent informativeness. It would also be interesting to introduce readability as an additional factor, perhaps by incorporating an \textit{Automated readability index}. Should the tool be made accessible to the public as a web application, there would also be the opportunity to use user rating of extracts to train a model for extract preference. The task of selecting readable, informative extracts that represent the cluster is an interesting task with various options for further work.

    Currently the approach models discussions as a flat list of posts --- without reply/response annotations. Using hierarchical discussion threads opens up interesting opportunities for Argument Mining using points extraction as a basis. A new summary section that listed points commonly made in response to other points in other posts would be a valuable addition. This would also be possible with discussions on \textit{Twitter} where reply information is also available. Related to this, adding user identity information would make it possible to track a posts by the same user in discussions and represents another interesting opportunity for further features.

    Another direction to explore would be to build a graph-like representation of the discussion. Using point's subject and object information, a graph of nodes representing nouns connected by verbs as edges could be generated. Semantic annotations from the verb frames could also be included. As part of this project we experimented with a presentation of this type, an example is included in Appendix \ref{app:disc_graph}. Related to this is the possibility of using the semantic annotations to investigate deeper abstractive summarization.

    There is also potential for further work on the summary presentation. Feedback gathered in Study 3 suggested that it would be useful to present the counts for counter points. This would allow the section to not only show a difference in opinion but to present how opinion is split. One comment suggested that the output was in between a quantitative and qualitative report and that adding more quantitative information to the summaries would make them more useful. Workshop attendees also suggested that being able to refer back to source text from the extracted points would also make the results more useful --- something again made possible with an online, interactive interface for summaries.

  \section{Conclusion}
    In this project we have shown that our points extraction is a viable foundation for summarization of online discussion. We think this success can be generalized to tasks beyond summarization that make use of text extracts. We have shown that existing tools struggle with the summarization of discursive text, and that our summaries with sections that are designed to give a more complete overview are preferred. Summarization is just one use case for such analysis and there are many other varied layers that could be built on top of our points extraction implementation.

    We see this project a step forward in the process of better understanding online discussion. Our hope is that this work can become the basis for further work and applications that allow the exploration of the wealth of ideas and arguments currently hidden in online discussions.
