\chapter{Research Goals\label{chap:res-goals}}
  Having justified the working on the topic, we now give an overview of our project's goals.

  First we wanted to develop a robust means of identifying and extracting points from text, this would serve as a foundation for later analysis. We wanted investigate not only if points were a viable content unit but if they could be better at succinctly presenting information than extracted sentences.

  We also were interested to expand on points to match points to counter points using negation and antonyms. Co-occurring points were also of interest. We also needed to find a way to present and test these. This led us to our final goal of using points in a summarization task. We wanted to investigate if points could be used to construct a summary that performed better than one based on sentence extracts. On top of this, could meta data about points extracted from text such as referenced topics, source and negation and be used to formulate a structured summary that readers find useful.

  Aware that the foundation analysis of extracting points could have utility beyond that of summarizing discussion, we wanted to build tools for analysis in a modular manner such that components may be reused in the future for further work.

  (I feel like this section is missing something but I don't know how to add more detail without going beyond the scope of the project or into the details of the method)
