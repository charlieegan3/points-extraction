\chapter{Summary Generation\label{chap:summary-generation}}
  A process to extract and refine a list of points could be used in a number of ways. From this foundation, we investigate a potential use for points in a summarization task. Extracted points are short and have a `pattern' that allows for new comparisons not possible with plain text extracts. This chapter first explains a number of shared components used in building each summary section, then goes on to describe each section in detail.

  \section{Permitted Extracts}
    In a cluster of points with the same patter there are a range of different extracts that could be selected. There is much variation in quality caused by additional words, bad parses and poor punctuation. Removing points with extracts that matched heuristics for poor quality would significantly reduce the information available to determine a cluster's salience. Such points are useful in aggregation but not for presentation.

    Instead we implemented a set of rules that prevent a point's extract from being used in the summary, even while the point itself remains in the cluster to aid aggregation and sorting tasks.

    Predominantly exclusions are made based on the presence of certain substrings using regular expressions. Exclusion patterns include: starts with a wh-question word or contains a question mark; has two or more consecutive words in block capitals or contains a mid-word case change. Since these patterns can be applied very quickly extracts are tested using these first.

    Following on from these, there are other more complex exclusion patterns based on the dependencies from re-parsing the extract using CoreNLP. Extracts with clausal or generic dependencies are excluded, such dependencies are characteristic of long points or erroneous parses. An extract may also be excluded if it contains more than two conjunctive relations or nominal dependencies.

    Finally some additional restrictions, that are not based on the dependencies or the strings, are applied. For example, an `it' must be preceded by \{on, and, but, whether\} and words are not allowed to be repeated.

  \section{Extract Selection}
    Points are manipulated in clusters where members share a common pattern, for example, there might be a cluster where each point had the pattern \texttt{woman.nsubj make.verb choice.dobj}. This cluster would contain all the points with this pattern and in turn all the extracts that could be used in a summary to express this point. Even after refining the list of points input to this component of the pipeline there is much variation in the quality of the extracts available for selection. Take this example cluster of extracts for points about the Genesis creation narrative:

    \begin{itemize}[label={}]
      \item{\blockquote{The world was created in six days.}}
      \item{\blockquote{The world was created in exactly 6 days.}}
      \item{\blockquote{Is there that the world could have been created in six days.}}
      \item{\blockquote{The world was created by God in seven days.}}
      \item{\blockquote{The world was created in 6 days.}}
      \item{\blockquote{But, was the world created in six days.}}
      \item{\blockquote{How the world was created in six days.}}
    \end{itemize}

    All of these passed the `Permitted Extracts' stage. The task is to select the best extract to represent the cluster. In this instance our approach selected the fifth point, \blockquote{The world was created in 6 days.}. This selection is made every time a cluster has been selected for use in a summary as an extract to represent it is required. This is done using a bag-of-words, length-weighted, bigram model of the extract words in the cluster. The goal was to select the most succinct extract that was most representative of the pattern cluster.

    Extracts are iterated and the bigrams for each collected. Bigrams are given a value equal to the number of times they occur in the cluster. Each extract is then given a value equal to the sum of the values for the bigrams it contains. This extract value is then divided by the number of words in the extract to determine a final score. The extract with the highest score is selected for use in the summary.

  \section{Extract Presentation \& Formatting}
    Our points extraction approach works by selecting the relevant components in a string for a given point, using the dependency parse graph. While this has a key advantage in creating shorter content units, it also means that extracts are often poorly formatted for presentation in isolation. To overcome this we needed to implement a means of correcting extracts for presentation, below are two examples that illustrate the problem with poor punctuation and style.

    \blockquote{that Scientifically , human life begins at this stage}
    and
    \blockquote{.that a fetus is a person ,}

    To overcome this we have a short function to ensure the extract meets the following properties: extract is capitalized; ends in a period; commas are not preceded by a space; contractions are applied where possible and consecutive punctuation marks condensed or removed. Certain determiners, adverbs and conjunctions (because, that, therefore) are also removed removed from \textit{the start} of extracts. The two extracts above are translated into the following cleaner versions:

    \blockquote{Scientifically, human life begins at this stage.}
    and
    \blockquote{A fetus is a person.}

    This particular feature still requires work. The task of correctly formatting a string has only been touched on here and is a candidate for further work on the project.

  \section{Avoiding Repetition in Summaries}
    A cluster's inclusion in a particular summary section is a function of the number of points in the cluster. This is based on the idea that larger clusters are have greater importance and should the highest priority to include in the summary. To avoid large clusters being repeatedly selected at each summary section, a list of used patterns and extracts is maintained.

    When an extract is used in a summary section it is `spent', and added to a list of used extracts. The extract pattern, string, lemmas and subject-verb-object triple are added to this list. Any point that matches anything in this list of used identifiers cannot be used. The same extract may be included in many clusters under different point patterns. For example, \blockquote{Life begins at conception.} is matched by both the \texttt{life.nsubj begins.verb} and \texttt{life.nsubj begins.verb at.prep conception.dobj} patterns. This means that when checking is an extract is suitable it must be checked against the text \textit{and} patterns of those already used.

    \tocless\subsection{Negation Shorthand}
      As part of the counter point analysis discussed in the following section, a condensed point representation was developed for expressing points that shared many of the same words. For example, to include both \blockquote{A fetus is a human} as well as \blockquote{A fetus is \textbf{not} a human} is a highly repetitive presentation. Such a presentation does not make for good reading and uses words that could be used to express additional information from another point.

      Based largely on a string diffing library\footnote{https://rubygems.org/gems/differ}, we developed an alternative, more condensed presentation for such pairs of points. Under this, the above examples could be written as \blockquote{A fetus is \textbf{\{}not\textbf{\}} a human}, saving five repeated words. However, more complex examples also occurred such as \blockquote{Abortion \textbf{\{}is not\textbf{|}should be\textbf{\}} legal \textbf{\{}after 12 weeks\textbf{\}}} (\blockquote{Abortion should be legal} and \blockquote{abortion is not legal after 12 weeks}). While there are adjustments that can be made to improve this, such as requiring the two extracts to have a similar number of words, complex examples still surfaced and the formatting was not used.

      The diffing functionality is still used in the identification of counter points.

  \section{Summary Topics}
    Topic words are used in the point extraction stage but are also passed along to the summarization stage. Topic words are used in some sections to avoid repetition. They are also used to highlight topic words in extracts in the formatted summary presentation.

  \section{Summary Sections}
    \subsection{Counter Points}
      The identification of counter points were a goal for the project from the start. Counter points are matched on one of two possible criteria, either the presence of negation terminology or a antonym for a work that is part of the point's pattern.

      Potential, antonym-derived counter points, for a given point, are generated using it's pattern. Taking the pattern \texttt{treatment.nsubj be.verb private.dobj} as an example, this generates the counter point \texttt{treatment.nsubj be.verb \textbf{public}.dobj}. These substitutions can be applied to any component that has an antonym, including verbs (\texttt{sperm.nsubj have.verbs rights.dobj} vs \texttt{sperm.nsubj \textbf{lack.verb} rights.dobj}). If there are many words in the pattern with antonym matches then multiple potential counter points are generated for a single point.

      Once a list of potential point vs. counter point patterns has been generated these are filtered to make sure that both the generated counter point pattern exists in the list of points and that the there are no duplicate pairs. A point/counter point pair cannot also appear in in reverse as part of another pair.

      Counter points are selected for display based on the average size of the clusters for the point and counter point. This is intended to represent ``the most common pairs of contradictory points''. In the summary text only the extract for the point is displayed, not the counter. The section is introduced as ``points that people disagree on'' so stating either point presents it as a point that is contested. Previously the `negation shorthand' was used to present points and counter points but variation in extracts often led to complex examples.

    \subsection{Negated Points}
      Negated points are displayed in the same summary section, `points that people disagree on'', however they are identified differently. Negated points are selected from the largest unused clusters. Negation terminology is not commonly part of the cluster pattern, for example, the \texttt{woman.nsubj have.verb right.dobj} cluster could include both \blockquote{A woman has the right} and \blockquote{The woman does not have the right} as extracts. Identifying these negated forms within clusters is how negation-derived counter points are collected.

      First the cluster is split into two groups, extracts with negation terminology and those without. The Cartesian product of these two groups gives all pairs of negated and non-negated extracts. For each of these pairs; a string difference is computed\footnote{https://rubygems.org/gems/differ}, the complexity of this difference pattern is used to identify a clean match. For example \blockquote{\{+that \}a fetus is \{-not \}a person\{- under the law of the eu\}} is not as clean as \blockquote{that a fetus is\{n't \}a person}. By counting elements in the difference pattern we are able to select shorter and cleaner examples where they exist.

      However, similarly to antonym-derived counter points, clean examples do not always exist. Negated points are represented as the positive side extract of the pair and listed under the ``points that people disagree on'' section.

    \subsection{Co-occurring Points}
      As well as opposing counter points we were also interested to present points that were commonly made by the same participant. Our corpus did not annotate participant's identity so the assumption was made that each post was from a unique participant. Adjusting the tool to work on discussion threads with hierarchical structure and participant identity would enable presentations such as `x is commonly said when responding to y''.

      To identify co-occurring points each post in the discussion was fire represented as a list of points it made. Taking all combinations of the points made in a post, for all posts, generates a list of all co-occurring points. Using the point patterns, these pairs can sorted based on the number of times they occur.

      Co-occurring pairs are rejected if they are too similar, patterns must differ by more than one component. For example, \texttt{woman.nsubj have.verb choice.dobj} could not be displayed as related to \texttt{woman.nsubj have.verb rights.dobj} but could be with \texttt{fetus.nsubj have.verb rights.dobj}.

      Pairs displayed in the summary can only use available points remaining after the counter point analysis. They must also have a suitable quality extract to display, if not the next most common pair may be used instead.
    \subsection{Commonly Occurring Points}
      Before selecting the points for the final sections of the summary, a number of the remaining most common points are listed. There are no restrictions on points from this section other than that they must not have been used by the previous sections. The intention is to present common points that are important to the discussion that have no counter or co-occurring point.

    \subsection{Topic Points}
      There was also an opportunity to display points broken down by topic. To determine the salient topics the subjects and objects for all clusters were tallied, this gave a ranking of topic words. This information is not available in the flat list of topics returned from the topic model.

      Using these common topics, points containing them can be selected and displayed under a section for that topic. Extracts for a given topic is selected based on the size of the cluster and but also minimizing repetition with the extracts already selected for that topic. Similarly, only points not used in the previous sections can be used.

    \subsection{Longer Pattern Points}
      Not all patterns are three components long. While patterns of three components represent most of the largest clusters, less common, longer form points also offer interesting extracts. Example extracts from longer form patterns include: \blockquote{The human life cycle begins at conception.} and \blockquote{I must not care about the mother.}. The patterns for these points also include the prepositional phrases which expand on the idea being discussed.

      Longer points are selected based on the size of the pattern, such a point must have more than three components. To avoid repetition, when longer form points are being listed, each consecutive point added to the section must introduce at least one new topic word. This requirement avoids common points differing only in the prepositional phrase being listed in this section.

    \subsection{Topic Linking Points}
      An alternative to selecting points points with a longer patter is to instead select points that mention more than one topic word. This allows less common extracts, that are still highly relevant based on their topic words, to be displayed.

      This section is based entirely on the point extracts, not the pattern clusters. Extracts are sorted on the number of topic words they include, the top 100 extracts are taken to form a cluster. From this cluster, the `Extract Selection' process is applied to repeatedly select the most presentable and representative extract. This is done until three `Topic Linking Points' are extracted for the section.

    \subsection{Common Questions Points}
      As a final idea for an informative summary section we opted to include a list of questions that had been asked a number of times. Interestingly repeated questions were far less common. To gather a list of common questions first the points with extracts containing a `?' were selected. These qualifying question extracts were then clustered using their patterns, extracts from up to the top three repeated questions were displayed. However, it was uncommon for there to be this many repeats.
