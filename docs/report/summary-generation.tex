\chapter{Summary Generation\label{chap:summary-generation}}
  A process to extract and refine a list of points could be used in a number of ways. From this foundation, we investigate a potential use case for points in a summarization task. Extracted points are short and have a `pattern' that enables new comparisons not possible with plain text extracts. One such comparison is the analysis of counter points. Argument analysis is performed within this module as it is closely linked to the task of generating summaries.

  This chapter first explains a number of utility components used in building each summary section, then goes on to describe how each section is generated in detail.

  \section{Permitted Extracts}
    In a cluster of points with the same pattern there are a range of different extracts that could be selected. There is much variation in quality caused by additional words, bad parses and poor punctuation. Completely removing points with extracts that matched heuristics for poor quality would significantly reduce the cluster size --- the key factor used in determining a cluster's salience. Such points are useful in analysis through aggregation but not for presentation in summaries.

    Instead we implemented a set of rules that prevent a point's extract from being used in the summary, even while the point itself remains in the cluster to aid aggregation and sorting tasks. \textit{This is fundamentally different from the curation task}. Points with valid patterns but poor quality extracts are still required in determining important clusters (member count information is used). Points excluded at the curation stage are not at all useful.

    Predominantly, points are prevented from being presented based on the presence of certain substring patterns tested with regular expressions. Exclusion patterns include: starts with a wh-question token or contains a question mark; has two or more consecutive words in block capitals or contains a mid-word case change. Since these patterns can be applied very quickly extracts are tested using these first before using more expensive analysis to make further exclusions.

    Following on from these, there are more complex exclusion patterns based on the dependencies from re-parsing the extract using CoreNLP. An extract may consist of many parts spread across the sentence; these do not always make sense. Extracts with clausal or generic dependencies (on re-parsing) are excluded. Such dependencies are characteristic of overly long extracts or erroneous parses. An extract may also be excluded from display if it contains more than two conjunctive relations or nominal dependencies.

    Finally some additional restrictions; not based on the dependencies or the strings; are applied. For example, an `it' must be preceded by \{on, and, but, whether\} and tokens are not allowed to be repeated.

  \section{Extract Selection}
    Points are manipulated in clusters where members share a common pattern, for example, there might be a cluster where each point had the pattern \texttt{woman.nsubj make.verb choice.dobj}. This cluster would contain all the points with this pattern and in turn all the extracts available for use in a summary to express this cluster's point. Even after refining the list of points input to this component of the pipeline, there is much variation in the quality of the extracts available for selection. Take this example cluster of extracts for points about the Genesis creation narrative:

    \begin{itemize}[label={}]
      \item{\blockquote{The world was created in six days.}}
      \item{\blockquote{The world was created in exactly 6 days.}}
      \item{\blockquote{Is there that the world could have been created in six days.}}
      \item{\blockquote{The world was created by God in seven days.}}
      \item{\blockquote{The world was created in 6 days.}}
      \item{\blockquote{But, was the world created in six days.}}
      \item{\blockquote{How the world was created in six days.}}
    \end{itemize}

    All of these passed the `Permitted Extracts' stage. The next task is to select the best extract to represent the cluster. In this instance our approach selected the fifth point, \blockquote{The world was created in 6 days.}. Selecting the best extract is performed every time a cluster (points with a pattern) has been selected for use in a summary. The best extracted is represent the cluster in human readable text. Selections are made using a bag-of-words, length-weighted, bigram model of the extract words in the cluster. The goal was to select the most succinct extract that was most representative of the entire cluster. We tested this in our evaluation, see Chapter \ref{chap:evaluation}.

    The bigrams for each each extract in the point cluster were collected. Bigrams were given a value equal to the number of times they repeat in the cluster. Each extract was then given a value equal to the sum of the values for the bigrams it contained. This aggregated score value was then divided by the number of words in the extract to determine a final score. The extract with the highest score was selected to present the cluster in the summary.

  \section{Extract Presentation \& Formatting}
    Our points extraction approach works by selecting the relevant components in a string for a given point, using the dependency parse graph. While this has a key advantage in creating shorter content units, it also means that extracts are often poorly formatted for presentation when viewed in isolation (not capitalized, leading commas etc.). To overcome this we needed to implement a means of correcting extracts for presentation. Two examples that illustrate the problem of poor punctuation and style are: \blockquote{\textit{that Scientifically , human life begins at this stage}} and \blockquote{\textit{.that a fetus is a person ,}}.

    To overcome such issues we have a function to ensure the extract meets the following properties: is capitalized; ends in a period; commas are not preceded by a space; contractions are applied where possible; and consecutive punctuation marks condensed or removed. Certain determiners, adverbs and conjunctions (because, that, therefore) are also removed removed from \textit{the start} of extracts. The two extracts above are translated into the following cleaner versions: \blockquote{\textit{Scientifically, human life begins at this stage.}} and \blockquote{\textit{A fetus is a person.}}.

    This relatively simple process does not use parse information and predominantly operates on the opening and closing characters of an extract. The intention is to quickly tidy up the string for display as a short sentence. This feature still requires work. The task of correctly formatting a string has only been touched on here and is a candidate for further work.

  \section{Avoiding Repetition in Summaries}
    A cluster's inclusion in a particular summary section is a function of the number of points in the cluster. This is based on the idea that larger clusters are of greater importance and should be first to feature. Accounting for repetition using this metric is commonplace in other automatic summarizers (often aimed at news data). However, when generating summaries of social media discussion it raises the question: will this result in only showing the majoritarian view? While this is likely, it is also hard to account for. We have a number of sections such as `points people disagree on' and `longer form points' that bring up less common points to partially overcome this issue. However, without stance annotated posts, it is hard to know that a summary is truly balanced. Currently the counts of points are the key factor in their selection for summary sections.

    To avoid large clusters being repeatedly selected at each summary section, a list of used patterns and extracts is maintained. When an extract is used in a summary section it is `spent' and added to a list of used patterns and extracts. The extract pattern, string, lemmas and subject-verb-object triple are added to this list. Any point that matches anything in this list of used identifiers cannot be used later in the summary. The same extract may be included in many clusters under different point patterns. For example, \blockquote{Life begins at conception.} is matched by both the \texttt{life.nsubj begins.verb} and \texttt{life.nsubj begins.verb at.prep conception.dobj} patterns. This means that when checking if a point is suitable it must be checked against the used extracts \textit{and} patterns of those previously selected.

    \tocless\subsection{Negation Shorthand}
      As part of the argument analysis of negated points (discussed in section \ref{neg-points}), a condensed point representation was developed for expressing extracts that shared many of the same words. For example, to include both \blockquote{A fetus is a human} as well as \blockquote{A fetus is \textbf{not} a human} is a highly repetitive presentation. Such a presentation does not make for good reading and uses surplus words that could be used to express additional information.

      Based largely on a string diffing library\footnote{https://rubygems.org/gems/differ}, we developed an alternative, more condensed presentation for such pairs of points. Under this, the above examples could be written as \blockquote{A fetus is \textbf{\{}not\textbf{\}} a human}, saving five repeated words. However, more complex examples also occurred such as \blockquote{Abortion \textbf{\{}is not\textbf{|}should be\textbf{\}} legal \textbf{\{}after 12 weeks\textbf{\}}} (\blockquote{Abortion should be legal} and \blockquote{abortion is not legal after 12 weeks}). While there are adjustments that can be made to improve this, such as requiring the two extracts to have a similar number of words, complex examples still surfaced and the formatting was eventually dropped from use in summaries.

      This diffing functionality is however still used when matching similar negated points into pairs (see section \ref{neg-points}).

  \tocless\subsection{Topic Words}
  Topic words are used in the point extraction stage are passed along to the summarization stage with the list of points (previously discussed in sections \ref{sec:lda-tect} \& \ref{sec:extract-process}). Topic words are used in some summary sections to guide selection and avoid repetition. They are also used to highlight topic words in extracts in the formatted summary presentation.

  \section{Summary Sections}
    A summary could be generated by listing the common points in the discussion. However, we were interested to explore beyond this into basic argumentation relations, i.e. point and counter points. We grouped summaries into a series of sections where each was the result of a different analysis on the list of extracted points. This section details how summaries were built up from these various different components.

    \subsection{Counter Points}
      The identification of counter points was a goal for the project from the start. This analysis was intended to highlight areas of disagreement in the discussion. Counter points are matched on one of two possible criteria, either the presence of negation terminology (see Section \ref{neg-points}) or an antonym for a pattern component.

      Potential, antonym-derived counter points, for a given point, are generated using its pattern and a list of antonyms. Antonyms were sourced from WordNet \cite{miller1995wordnet} using the NLTK corpus reader interface\footnote{http://www.nltk.org/howto/wordnet.html}. Taking the pattern \texttt{woman.nsubj have.verb right.dobj} as an example, this generates the following counter points:

      \begin{itemize}[label={$\bullet$}]
        \item{\texttt{man.nsubj \textcolor{gray}{have.verb right.dobj}}}
        \item{\texttt{\textcolor{gray}{woman.nsubj} lack.verb \textcolor{gray}{right.dobj}}}
        \item{\texttt{\textcolor{gray}{woman.nsubj have.verb} left.dobj}}
      \end{itemize}

      This shows that substitutions can be applied to any component that has an antonym, including verbs. If there are many words in the pattern with antonym matches then multiple potential counter points are generated for a single point.

      Once a list of potential point vs. counter point patterns has been generated, these are filtered to ensure that the generated counter point pattern exists in the list of points. From the example above, only the first generated pattern: \texttt{man.nsubj have.verb right.dobj} appeared in our Abortion discussion. A check is also made to remove duplicate pairs - a point/counter point pair cannot also appear in reverse order. Counter points must be three or more components in length, e.g. the pair \texttt{life.nsubj begin.verb} and \texttt{life.nsubj end.verb} cannot occur. Points with shorter patterns tend to have more varied extracts and lead to `counter points' that are not suitably matched. Pairs generated from the verbs `come' and `go' are discarded as they also lead to low-quality matches.

      These pairs are selected for display based on the average cluster size for the point and counter point. This is intended to represent ``the most common pairs of contradictory points''. In the summary text, only the extract for the point is displayed, not the counter. The section is introduced as ``points that people disagree on'' so stating either presents it as a point that is contested. Previously the `negation shorthand' was used to present points and counter points but variation in extracts often led to complex examples that were hard to read.

    \subsection{Negated Points \label{neg-points}}
      Negated points are displayed in the same summary section, ``points that people disagree on'', however they are identified differently. Negated points are selected from the largest clusters that have not already been used in the generation of previous sections. Negation terminology is not commonly part of the cluster pattern, for example, the \texttt{woman.nsubj have.verb right.dobj} cluster could include both \blockquote{A woman has the right} and \blockquote{The woman does not have the right} as extracts. Identifying these negated forms \textit{within} clusters is how negation-derived counter points are collected.

      First the cluster is split into two groups, extracts with negation terminology and those without. The Cartesian product of these two groups gives all pairs of negated and non-negated extracts. For each of these pairs a string difference is computed\footnote{https://rubygems.org/gems/differ}, the complexity of this difference pattern is used to identify a clear match. For example \blockquote{\{+that \}a fetus is \{-not \}a person\{- under the law of the eu\}} is not as clear as \blockquote{that a fetus is\{n't \}a person}. By counting elements in the difference pattern we are able to select shorter and cleaner examples where they exist.

      However, similarly to antonym-derived counter points, clean examples do not always exist. Negated points are represented a single extract from the pair and listed under the same \blockquote{\textit{points that people disagree on}} section.

    \subsection{Co-occurring Points}
      As well as counter points we were also interested to present points that were commonly made by the same participant. These related pairs were used in a summary section intended to show points that were positively related, i.e. raised in conjunction with one another, rather than being negated or contradictory. Our corpus did not annotate participant's identities so the assumption was made that each post was from a unique participant.

      To identify co-occurring points, each post in the discussion was first represented as a list of points it made. Taking all pairwise combinations of the points made in a post, for all posts, generates a list of all co-occurring points. Using the point patterns, these pairs can be sorted based on the number of times they occur.

      Co-occurring pairs are rejected if they are too similar --- patterns must differ by more than one component. For example, \texttt{woman.nsubj have.verb choice.dobj} could not be displayed as related to \texttt{woman.nsubj have.verb rights.dobj} but could be with \texttt{fetus.nsubj have.verb rights.dobj}.

      Pairs displayed in the summary can only use available points remaining after the counter point analysis. They must also have a suitable quality extract to display, if not the next most common pair is used instead.
    \subsection{Commonly Occurring Points}
      Before selecting the points for the remaining sections of the summary, a number of the remaining most common points are listed. There are no restrictions on points from this section other than that they must not have been used by the previous sections. The intention is to present common points important to the discussion that have no counter or co-occurring point.

    \subsection{Topic Points}
      There was also an opportunity to display points broken down by topic. To determine the salient topics the subjects and objects for all clusters were tallied. This gave a ranking of topic words used in points. This information is not available in the unordered list of topics returned from the topic model that is passed along the pipeline with the extracted points.

      Using these common topics, points containing them can be selected and displayed in a dedicated section for that topic. Extracts for a given topic are selected based on the size of the cluster and while also minimizing repetition with the extracts already selected for that topic. Similarly, only points not used in the previous sections can be used.

    \subsection{Longer Pattern Points}
    Not all patterns are three components long. While patterns of three components represent most large clusters, less common, longer form points offer more developed extracts. Example extracts from longer form patterns include: \blockquote{\textit{The human life cycle begins at conception.}} and \blockquote{\textit{A human being refers to a specific living organism of a specific kind.}}. The patterns for these points also include the prepositional phrases which expand on the idea being discussed.

      Longer points are selected based on the number of components in the pattern; such points must have more than three components. To avoid repetition, when longer form points are being listed, each consecutive point added to the section must introduce at least one new topic word. This requirement avoids common points differing solely on the prepositional phrase being listed in this section.

    \subsection{Topic Linking Points}
      An alternative to selecting points points with a longer pattern is to instead select points that mention more than one topic word. This allows less common extracts, that are still relevant, to be displayed.

      This section is based entirely on the point extracts, not the pattern clusters. Extracts are sorted on the number of topic words they include. From this sorted list, the top 100 extracts are taken to form a group. Next, the previously described `Extract Selection' process is applied to repeatedly select the most presentable and representative extract. This is done until three `Topic Linking Points' are extracted for the section.

    \subsection{Common Questions Points}
      As a final idea for an informative summary section we opted to include a list of questions that had been asked a number of times, repeated questions were far less common. To gather a list of common questions we started by selecting points with extracts containing question marks. These question points were then clustered using their patterns. While extracts from up to the top three repeated questions were displayed, it was uncommon for there to be three repeated questions to complete the section.
