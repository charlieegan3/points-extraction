\documentclass{article}
\usepackage[a4paper,margin=1cm]{geometry}
\usepackage[dvipsnames]{xcolor}
\usepackage{listings}
\usepackage{minted}

\pagenumbering{gobble}

\title{Project Source Listing}
\date{}

\begin{document}
  \maketitle
  This document lists all important project files. Files and folders are described in the maintenance manual.
  \pagebreak


\section*{docker-compose.yml}
\begin{minted}[linenos=true,breaklines=true]{YAML}
analysis_api:
  container_name: analysis_api
  build: ./analysis_api
  links:
    - points_api:points_api
    - topic_api:topic_api
  volumes:
    - "./analysis_api:/app"

curator:
  container_name: curator
  build: ./curator
  volumes:
    - "./curator:/app"

summarizer:
  container_name: summarizer
  build: ./summarizer
  links:
    - corenlp_server:corenlp_server
  volumes:
    - "./summarizer:/app"

topic_api:
  container_name: topic_api
  build: ./topic_api
  environment:
    PORT: 4567
  ports:
    - "4567:4567"
  volumes:
    - "./topic_api:/app"

points_api:
  container_name: points_api
  build: ./points_api
  environment:
    PORT: 4567
    CNLP_PORT: 9000
  ports:
    - "4568:4567"
  links:
    - corenlp_server:corenlp_server
    - neo4j:neo4j
  volumes:
    - "./points_api:/app"
  restart: unless-stopped

neo4j:
  container_name: neo4j
  image: neo4j
  environment:
    NEO4J_AUTH: none
  ports:
    - "7474:7474"
  mem_limit: "2g"
  memswap_limit: "2g"

corenlp_server:
  container_name: corenlp_server
  build: ./corenlp_server
  ports:
    - "9000:9000"
  environment:
    RELEASE: "stanford-corenlp-full-2015-12-09"
    RELEASE_URL: "http://nlp.stanford.edu/software/stanford-corenlp-full-2015-12-09.zip"
  mem_limit: "2g"
  memswap_limit: "2g"


\end{minted}
\pagebreak

\section*{analysis\_api/clean.rb}
\begin{minted}[linenos=true,breaklines=true]{Ruby}
# clean.rb
#
# This file loads each post in a corpus in turn, ensures that it has valid
# encoding, parses any metadata, and saves it as a json file

require 'json'

unless ARGV[0]
  puts "missing corpus argument"
  puts "try: ruby clean.rb abortion"
  exit
end

# for each of the posts in the corpus
Dir.glob("#{ARGV[0]}/*").to_a.each do |file, index|
  puts file

  # remove json files previously generated by this file
  if file.include? "json"
    `rm #{file}`
    next
  end
  content = File.open(file).read

  # reject invalid characters that cause issues during point extraction
  content = content.chars.select(&:valid_encoding?).join

  # parse metadata lines (#key=value)
  meta = Hash[*content.scan(/^#.*=.*$/).map { |e| e.delete('#').split("=") }.flatten]
  meta = meta.inject({}){ |memo,(k,v)| memo[k.to_sym] = v; memo }.merge!(post: file.gsub(/[^0-9]/, ''))

  # remove invalid substrings such as URLs
  content = content
    .gsub(/^#.*=.*$/, "")
    .gsub(/\[[0-9]+\]/, "")
    .gsub(/http\S+|\S{30,}/, "")
    .gsub(/([a-z]{3,}\.)([A-Z][a-z]{3,})/, ' ')
    .gsub(/([a-z]{3,}\.)([0-9]\.)/, ' ')
    .gsub(/\s+/, " ")
    .gsub("[...]", "")
    .gsub("/>", "")
    .gsub(" / ", "")
    .strip

  # output the file as json using the same name, only with a .json extension
  processed_file = meta.merge(content: content)
  File.open("#{file}.json", "w").write(processed_file.to_json)
end


\end{minted}
\pagebreak

\section*{analysis\_api/collector.rb}
\begin{minted}[linenos=true,breaklines=true]{Ruby}
# collector.rb
#
# This requests points for each post in a corpus, corpus name (folder) is
# passed as an argument

require "json"
require 'net/http'

unless ARGV[0]
  puts "missing corpus argument"
  puts "try: ruby collector.rb abortion"
  exit
end

path = "#{ARGV[0]}/*"
puts "Path: #{path}"
posts = []

# aggregate all the posts in the corpus
Dir.glob(path) do |f|
  next unless f.include? "json"
  post = JSON.parse(File.open(f).read)
  posts << post if post["content"].length > 30
end

# collect text from the corpus to use in modeling topics
topic_text = posts.map { |p| p["content"] }.join("\n").gsub(/[^\w']/, " ").gsub(/\s+/, " ").downcase[0..60000]

# setup a connection to the topic API
uri = URI('http://topic_api:4567/')
http = Net::HTTP.new(uri.host, uri.port)
topic_query = { text: topic_text, topic_count: 8, top_word_count: 8 }.to_json

# Request the topics from the topic API
req = Net::HTTP::Post.new(uri)
req.body = topic_query
topics = JSON.parse(http.request(req).body)["topics"]

# Create the output file for topics and points
out_file = File.open("#{ARGV[0]}_points.txt", "w")

out_file.write("#{topics.join(",")}\n")

# Setup a connection to the points API
uri = URI('http://points_api:4567/')
http = Net::HTTP.new(uri.host, uri.port)

# for each post, extract the points using the points API
posts.each_with_index do |post, index|
  query = { text: post["content"], topics: topics, keys: %w(string pattern) }.to_json
  req = Net::HTTP::Post.new(uri)
  req.body = query
  data = JSON.parse(http.request(req).body)
  # write the result of each to file
  data.map { |p| out_file.write("#{p.merge(post).to_json},\n") }
end


\end{minted}
\pagebreak

\section*{points\_api/lib/corenlp\_client.rb}
\begin{minted}[linenos=true,breaklines=true]{Ruby}
# corenlp_client
#
# This is a wrapper used in requesting dependency parse information from
# the CoreNLP service.

class CoreNlpClient
  def initialize(host)
    #params = URI.encode('properties={"annotators": "lemma,parse,depparse", "parse.flags": " -makeCopulaHead"}')
    params = URI.encode('properties={"annotators": "lemma,tokenize,ssplit,depparse"}')

    @uri = URI(host + "/?" + params)
    @http_client = Net::HTTP.new(@uri.host, @uri.port)
  end

  def request_parse(text)
    req = Net::HTTP::Post.new(@uri)
    req.body = text
    JSON.parse(@http_client.request(req).body)['sentences'].map do |sentence|
      [sentence['tokens'], sentence['collapsed-ccprocessed-dependencies']]
    end
  end
end


\end{minted}
\pagebreak

\section*{points\_api/lib/frame.rb}
\begin{minted}[linenos=true,breaklines=true]{Ruby}
# frame.rb
#
# This class represents a frame. Frames loads from the index are used to
# initialize instances of this class. This exposes a number of helper methods
# used in processing frames and points.

class Frame
  attr_accessor :components, :pattern_string, :source, :example, :verb
  def initialize(raw_frame, verb)
    @pattern_string = raw_frame['syntax'].map do |c|
      c['value'] ? "#{c['name']}.#{c['value']}" : c['name']
    end.join(" ")
    @components = components_for_pattern
    @source = raw_frame['source']
    @copula = raw_frame['copula']
    @example = raw_frame['examples'].first
    @verb = verb
  end

  def is_copula?
    @copula == true
  end

  def pos_pattern_string
    @components.map { |c| c[:pos] }.join(" ")
  end

  def components_for_pattern
    [].tap do |components|
      @pattern_string.split(/\s+/).each_with_index do |subpattern, index|
        components << build_component(subpattern, index)
      end
    end
  end

  def build_component(subpattern, index)
    tags = subpattern.scan(/(?:[A-Z]+)([\.\-_][\w-]+)/).flatten
    pos = subpattern.dup
    tags.each { |t| pos.gsub!(t, '') }
    tags.map! { |t| t.gsub(/^[\W_]+|[\W_]+$/, '') }
    return { index: index, string: subpattern, pos: pos, tags: tags }
  end

  def to_hash
    {
      string: @pattern_string,
      bare_frame: pos_pattern_string,
      missing_representation: 'not implemented',
      source: @source,
      example: @example,
      verb: @verb,
    }
  end

  def self.universal_frame(verb)
    { "examples"=>["any verb pattern"],
      "syntax"=> [{ "name"=>"VERB", "value"=>"Universal" }] }
  end

  def self.copula_frames(copula_verb)
    [
      {
        "examples"=>["He was a builder."], "copula"=>true,
        "syntax"=> [
          {"name"=>"NP", "restrictions"=>[]},
          {"name"=>"VERB", "value"=>"Copula", "restrictions"=>[]},
          {"name"=>"NP", "restrictions"=>[]}
        ]
      },
      {
        "examples"=>["He was fat."], "copula"=>true,
        "syntax"=> [
          {"name"=>"NP", "restrictions"=>[]},
          {"name"=>"VERB", "value"=>"Copula", "restrictions"=>[]},
          {"name"=>"ADJ", "restrictions"=>[]}
        ]
      },
      {
        "examples"=>["The rope came loose"], "copula"=>true,
        "syntax"=> [
          {"name"=>"NP", "restrictions"=>[]},
          {"name"=>"VERB", "value"=>"Copula", "restrictions"=>[]},
          {"name"=>"ADV", "restrictions"=>[]}
        ]
      }
    ].map { |f| Frame.new(f, copula_verb) }
  end
end


\end{minted}
\pagebreak

\section*{points\_api/lib/neo4j\_client.rb}
\begin{minted}[linenos=true,breaklines=true]{Ruby}
# neo4j_client.rb
#
# This is a wrapper for the Neo4j services and is used to execute queries
# and fetch results using a simplified interface.

class Neo4jClient
  def initialize(url)
    Neo4j::Session.open(:server_db, url)
  end

  def sentence_query_components(tokens, dependencies, sentence_index)
    nodes = tokens.map do |t|
      [t['word'], t['pos'], t['lemma'], t['index'].to_i-1]
    end
    relations = dependencies.uniq.map do |d|
      [d['dep'], d['governor'].to_i-1, d['dependent'].to_i-1]
    end
    relations.reject! { |l, g, d| g == d || l == 'ROOT'}

    return if relations.empty?

    nodes.map! do |word, pos, lemma, index|
      options = { word: word, part_of_speech: pos, lemma: lemma, index: index }
      Node.string_for_create(options, sentence_index)
    end
    query_string = "#{nodes.join(", ")}"

    relations.map! do |label, parent_index, child_index|
      Relation.string_for_create("n#{parent_index}", label, "n#{child_index}", sentence_index)
    end
    query_string += ", #{relations.join(", ")}"
  end

  def generate_create_query_for_sentences(sentences)
    query_string = "CREATE " + sentences.each_with_index.to_a.map do |data, index|
      sentence_query_components(*data, index)
    end.compact.join(", ")
  end

  def execute(query_string)
    Neo4j::Session.query(query_string)
  end

  def clear
    Node.delete_all
  end

  def verbs
    expl_verb_query = %q{MATCH (expl_verb:Node) WHERE expl_verb.part_of_speech =~ 'VB.?'
                         MATCH (expl_verb)-[r:REL]-(expl:Node)
                         WHERE r.label = "expl"
                         RETURN DISTINCT expl_verb.uuid as uuid;}
    expl_verb_uuids = Neo4j::Session.query(expl_verb_query).map(&:uuid)

    non_aux_verb_query = %q{MATCH (verb:Node) WHERE verb.part_of_speech =~ 'VB.?'
                            MATCH p=(verb)--(x)
                            WHERE NOT ANY (r in relationships(p) WHERE r.label =~ 'aux.*')
                            RETURN DISTINCT verb;}
    Neo4j::Session.query(non_aux_verb_query)
      .map(&:verb)
      .reject { |v| expl_verb_uuids.include? v.uuid }
  end

  def permitted_descendants(node, copula)
    standard_query = %q{match (verb:Node {uuid: "NODE_UUID"})
                        match p=(verb)-[*]->(related)
                        where NOT ANY (l IN ['advcl', 'csubj', 'ccomp', 'dep', 'parataxis'] WHERE ANY (r IN relationships(p) WHERE r.label =~ l))
                        return verb, related;}
    copula_query = %q{match (verb:Node {uuid: "NODE_UUID"})
                      match (cop:Node)-[rel_cop:REL]->(verb)
                      match p=(cop)-[*]->(related)
                      where rel_cop.label = "cop"
                      and NOT ANY (l IN ['advcl', 'csubj', 'ccomp', 'dep', 'parataxis'] WHERE ANY (r IN relationships(p) WHERE r.label =~ l))
                      and related <> verb
                      return verb, cop, related;}
    if copula
      [standard_query, copula_query].map do |query|
        query.gsub!('NODE_UUID', node.uuid)
        Neo4j::Session.query(query).to_a
      end.max_by(&:size)
    else
      return Neo4j::Session.query(standard_query.gsub('NODE_UUID', node.uuid)).to_a
    end
  end

  def permitted_descendant_string(node, copula=false)
    permitted_descendants(node, copula).to_a.map do |result|
      result.to_h.to_a.map(&:last)
    end.flatten.uniq.sort_by(&:index).map(&:word).join(" ")
  end
end


\end{minted}
\pagebreak

\section*{points\_api/lib/node.rb}
\begin{minted}[linenos=true,breaklines=true]{Ruby}
# node.rb
#
# This class extends the Neo4j::ActiveNode class and defines the structure that
# should be used to save tokens in the Neo4j database. This includes attributes
# and dependency relations to other nodes.

class Node
  include Neo4j::ActiveNode

  property :word
  property :part_of_speech
  property :lemma
  property :index

  has_many :out, :children, model_class: :Node, rel_class: :Relation
  has_many :in, :parents, model_class: :Node, rel_class: :Relation

  def self.string_for_create(options, sentence_index)
    options.merge!(uuid: SecureRandom.uuid)
    attribute_string = options.map do |k,v|
      v = "\"#{v}\"" unless v.is_a? Integer
      "#{k} : #{v}"
    end.join(", ")

    "(s#{sentence_index}n#{options[:index]}:Node {#{attribute_string}})"
  end
end


\end{minted}
\pagebreak

\section*{points\_api/lib/points\_extraction.rb}
\begin{minted}[linenos=true,breaklines=true]{Ruby}
# points_extraction.rb
#
# This is the central file to extracting points from text. This is called from
# the application twice, first to get the verbs that have valid frame matches
# from in the database and then, for each of these, to get the extract
# information for the match as a complete point, ready to send in the response.

module PointsExtraction
  COPULAE = %w(act appear be become come end get go grow fall feel keep look prove remain run seem smell sound stay taste turn wax)

  # This will complete the points extraction process, given that some verbs
  # matched frames, this will complete the point structure to return
  def self.points_for_matches(neo4j_client, matches, topics, keys)
    points = matches.map do |verb, results|
      results.reject! { |r| r.first.to_a.compact.size < 2 }
      next if results.empty?
      string = neo4j_client.permitted_descendant_string(verb, PointsExtraction::COPULAE.include?(verb.lemma))
      results.map do |r|
        match = Hash[r.first.to_h.to_a.reject { |e| e.last.nil? }]
        if r.last.map { |f| f[:frames] }.uniq == [:generic] &&
             r.first.to_a.compact.size < 3 &&
             (match.map {|_, v| v.word } & topics).empty? &&
             (match.map {|_, v| v.word }.map(&:length).reduce(&:+) / match.size) < 5
          next
        end
        result = {}
        result.merge!(string: string) if keys.include? 'string'
        result.merge!(match: match) if keys.include? 'match'
        result.merge!(pattern: match.map { |k, v| "#{v.lemma}.#{k}" }.join(" ")) if keys.include? 'pattern'
        result.merge!(frames: r.last.map { |f| f[:frames] }.uniq) if keys.include? 'frames'
        result
      end
    end.flatten.compact
  end

  # this iterates all the verbs and tests that they each match frames, those
  # that do are returned
  def self.matches_for_verbs(neo4j_client, frames, frame_queries)
    verb_queries = neo4j_client.verbs.map do |v|
      queries_for_verb(v, frames, frame_queries).map do |q|
        { uuid: v.uuid, query: q.first, frames: q.last }
      end
    end.flatten.group_by { |e| e[:query] }

    verb_queries.map do |shared_query, frame_queries|
      shared_query = shared_query.gsub("UUIDS", frame_queries.map { |q| q[:uuid] }.join('", "'))
      neo4j_client.execute(shared_query).to_a.map { |r| [r, frame_queries] }
    end.flatten(1).compact.group_by { |r| r.first.verb }.reject { |_, v| v.empty? }
  end

  private
  # this fetches the cql queries for the frames
  def self.queries_for_verb(verb, frames, frame_queries)
    return [] unless frames[verb.lemma]
    frames = frames[verb.lemma].map { |f| Frame.new(f, verb.lemma) }
    if COPULAE.include? verb.lemma
      frames += Frame.copula_frames(verb.lemma)
    end
    [].tap do |queries|
      frames.group_by { |f| [f.pos_pattern_string, f.is_copula?] }.each do |pattern, frames|
        query_name = pattern.last ? pattern.first + '-cop' : pattern.first
        queries << [frame_queries[query_name], frames] if frame_queries[query_name]
      end
    end.push([frame_queries['VERB-UNIVERSAL'], :generic])
  end
end


\end{minted}
\pagebreak

\section*{points\_api/lib/relation.rb}
\begin{minted}[linenos=true,breaklines=true]{Ruby}
# relation.rb
#
# This class extends the Neo4j::ActiveRel class and defines the structure of a
# dependency graph edge. These objects only have a single label attribute and
# connect two tokens nodes in the graphs.

class Relation
  include Neo4j::ActiveRel

  from_class :Node
  to_class   :Node
  type 'REL'

  property :label

  def self.string_for_create(from, relation, to, sentence_index)
    "(s#{sentence_index}#{from})-[s#{sentence_index.to_s+from+to+SecureRandom.hex[0..5]}:REL { label : \"#{relation}\" }]->(s#{sentence_index}#{to})"
  end
end


\end{minted}
\pagebreak

\section*{points\_api/lib/utils.rb}
\begin{minted}[linenos=true,breaklines=true]{Ruby}
# utils.rb
#
# This implements a series of helper functions. Used in the extraction  of points


module Utils
  def self.chunk_text(chunk_length, text)
    text.gsub!(/([a-z]\.)([A-Z])/, ' ')
    lines = text.split(/$|\. /)

    things=[].tap do |chunks|
      chunks << ""
      until lines.empty? do
        chunks << "" if (chunks.last + lines.first).length > chunk_length
        chunks[-1] += ". " + lines.shift
      end
    end
  end

  def self.clean_text(text)
    text.encode(Encoding.find('UTF-8'), { invalid: :replace, undef: :replace, replace: ''})
        .gsub(/[^\w\s\n\.,\(\)\{\}\]\["'\$£;:\-&\?]/, " ").gsub('"', '')
  end

  def self.sentence_contains_topic(sentence, topics)
    string = sentence.first.map { |t| t['word'] }.join.downcase
    topics.each do |t|
      return true if string.include? t
    end
    false
  end
end


\end{minted}
\pagebreak

\section*{points\_api/tasks/parse\_verb\_net.rb}
\begin{minted}[linenos=true,breaklines=true]{Ruby}
# parse_verb_net.rb
#
# This is a script used in parsing the XML information from VerbNet. VerbNet
# is not stored in it’s original form in the project source but is available
# for download here: http://verbs.colorado.edu/verb-index/vn/verbnet-3.2.tar.gz

require 'json'
require 'nokogiri'
require 'pry'

verbs = Hash.new([])

Dir.glob('verb_frames/*.xml') do |path|
  doc = Nokogiri::XML(File.open(path).read)
  members = doc.css('MEMBERS MEMBER').map { |e| e['name'] }

  frames = doc.css('FRAMES FRAME').map do |frame|
    examples = frame.css('EXAMPLE').map(&:text)
    syntax = frame.css('SYNTAX > *').map do |component|
      { name: component.name, value: component['value'],
        restrictions: component.css('SYNRESTR').map{ |s| { value: s['value'], type: s['type'] } }}
    end
    semantics = frame.css('SEMANTICS > *').map do |predicate|
      {
        value: predicate['value'],
        args: predicate.css('ARG').map { |a| { type: a['type'], value: a['value'] } }
      }
    end
    { examples: examples, syntax: syntax, semantics: semantics, source: path }
  end

  members.each do |member|
    verbs[member] += frames
  end
end

File.open('verbs.json', 'w').write(verbs.to_json)
puts "Wrote #{verbs.size} to verbs.json"


\end{minted}
\pagebreak

\section*{stock\_summarizers/summarizer\_topic.py}
\begin{minted}[linenos=true,breaklines=true]{Python}
#!/usr/bin/env python

# summarizer_topic.py
# This is the implementation of the ‘Stock’ summarizer used in the evaluation
# that was adjusted to run as part of the summary generation pro- cess.
# Originally sourced from:
# http://homepages.abdn.ac.uk/advaith/pages/teaching/NLP/practicals/Practical3.zip

import re, random, math, collections, itertools, sys, os

#------------- Function Definitions ---------------------

#calculates p(W)
def getProbabilities(sentences, pWord):
    freq = {} # {} initialises a dictionary [hash function]
    allWordsTot = 0
    #iterate through each sentence
    for sentence in sentences:
        wordList = re.findall(r"[\w']+", sentence)

        for word in wordList: #iterate over words in sentence
            allWordsTot += 1 # keeps count of total words in dataset
            if not (word in freq):
                freq[word] = 1
            else:
                freq[word] += 1


        # Calculate p(word)
    for word in freq.keys():
        pWord[word] = (freq[word] / float(allWordsTot) )

def cleanSentence(sentence):
    sentence = re.sub('^\W+', '', sentence)
    sentence = re.sub('[^\w\.\?]+$', '', sentence)
    sentence = re.sub('\/>?', '', sentence)
    sentence = re.sub('\.+', '.', sentence)
    sentence = re.sub('\s+', ' ', sentence)
    return sentence.capitalize()

def wordList(sentence):
    return re.findall(r"[\w']+", sentence)

def wordCount(sentence):
    return len(wordList(sentence))

#----------------------------------------------------------

def scoreSentences(sentences, pWord, maxLength):
    summaryLength = 0
    iters = 0
    while summaryLength <= maxLength and iters < 1000:
        scores = {}
        iters += 1

        sentences = [s for s in sentences if (len(s) < (maxLength * 1.05) - summaryLength)]
        sentences = [s for s in sentences if (wordCount(s) > 3)]
        if len(sentences) == 0:
            break

        for s in sentences:
            words = wordList(s)
            count = len(words)
            score = 0
            for w in words:
                score += pWord[w]
            scores[s] = score / count

        s = max(scores, key=scores.get)

        if summaryLength <= (maxLength * 1.05):
            print (cleanSentence(s))
            summaryLength += wordCount(s)
        for word in wordList(s): #iterate over words and reduce probabilities
            pWord[word] *= pWord[word]

#-----------------------------------------------------------

f = sys.argv[1]
f = open("../stock_summarizers/" + f + ".txt", 'r')

text = f.read().replace('#stance=stance1', '').replace('#stance=stance2', '')
text = ''.join([i if ord(i) < 128 else '' for i in text])
#text = text.replace('.', '. ')

#sentences = re.split(r'\n|\. ', text)
sentences = re.split(r'\n', text)

pWord={}
getProbabilities(sentences,  pWord)

scoreSentences(sentences, pWord, int(sys.argv[2]))


\end{minted}
\pagebreak

\section*{summarizer/abortion\_formatted.html}
\begin{minted}[linenos=true,breaklines=true]{HTML}
<p>People <strong>disagree</strong> on these points:</p>
        <blockquote>
      <p><code>Abortion</code> should be illegal.</p>
    </blockquote>
          <blockquote>
      <p><code>Life</code> begins at conception.</p>
    </blockquote>
          <blockquote>
      <p>I am <code>pro</code>-<code>life</code> for the most part.</p>
    </blockquote>
  
  <blockquote>
    <p>A <code>fetus</code> is a <code>person</code>.</p>
  </blockquote>
  <blockquote>
    <p>Animals and fetuses have <code>rights</code>.</p>
  </blockquote>
  <blockquote>
    <p><code>Abortion</code> is still murder.</p>
  </blockquote>

<p><strong>Commonly occurring points</strong> made in the discussion were:</p>
  <blockquote>
    <p>Seriously : I am against all <code>abortion</code>.</p>
  </blockquote>
  <blockquote>
    <p>She is pregnant and the <code>baby</code>.</p>
  </blockquote>
  <blockquote>
    <p>Taking the <code>life</code> of another <code>person</code>.</p>
  </blockquote>

<p>Users that talk about <strong><em>X</em> often talk about <em>Y</em></strong>.</p>
  <blockquote>
    <p>
    <strong>(<em>X</em>)</strong>
      You have no right to any <code>life</code>.      <strong>(<em>Y</em>)</strong>
      Not to have an <code>abortion</code>.    </p>
  </blockquote>
  <blockquote>
    <p>
    <strong>(<em>X</em>)</strong>
      She can't have an <code>abortion</code>.      <strong>(<em>Y</em>)</strong>
      <code>Abortion</code> should not be banned.    </p>
  </blockquote>
  <blockquote>
    <p>
    <strong>(<em>X</em>)</strong>
      Otherwise a <code>person</code> will die.      <strong>(<em>Y</em>)</strong>
      Save the <code>life</code> of someone.    </p>
  </blockquote>

<p>Common points made in the discussion that <strong>link terms</strong> were:</p>
  <blockquote>
    <p>A <code>human</code> being refers to a specific <code>living</code> organism of a specific kind.</p>
  </blockquote>
  <blockquote>
    <p>The <code>human</code> <code>life</code> cycle begins at conception.</p>
  </blockquote>
  <blockquote>
    <p>As when <code>life</code> begins for a butterfly or moth.</p>
  </blockquote>

<p>Points for <strong>commonly discussed topics</strong>:</p>
<ul>
      <li>
      <p>Abortion              <blockquote>
          <p>Seeking an early, first term <code>abortion</code>.</p>
        </blockquote>
              <blockquote>
          <p>Banning partial birth <code>abortion</code>.</p>
        </blockquote>
              <blockquote>
          <p>Get an <code>abortion</code> in no way.</p>
        </blockquote>
          </li>
      <li>
      <p>Woman              <blockquote>
          <p>You are a very brave and strong <code>woman</code>.</p>
        </blockquote>
              <blockquote>
          <p>Saving to the pregnant <code>woman</code>.</p>
        </blockquote>
              <blockquote>
          <p>After a <code>woman</code> has an <code>abortion</code>.</p>
        </blockquote>
          </li>
      <li>
      <p>Fetus              <blockquote>
          <p>Killing an innocent <code>fetus</code>.</p>
        </blockquote>
              <blockquote>
          <p>They are fetuses, not <code>human</code> beings.</p>
        </blockquote>
              <blockquote>
          <p>All of us were once a <code>fetus</code>.</p>
        </blockquote>
          </li>
  </ul>

<p>Points about <strong>multiple topics</strong>:</p>
  <blockquote>
    <p>Find the balance between a womans <code>choice</code> and protecting <code>human</code> <code>life</code>.</p>
  </blockquote>
  <blockquote>
    <p><code>Abortions</code> <code>kill</code> <code>life</code>.</p>
  </blockquote>
  <blockquote>
    <p><code>Abortion</code> is not killing a <code>human</code> <code>child</code>.</p>
  </blockquote>

<p>People ask <strong>questions</strong> like:</p>
  <blockquote>
    <p>When does <code>HUMAN</code> <code>life</code> begin?</p>
  </blockquote>
  <blockquote>
    <p>Do you value your <code>life</code> above that of 3,613 strangers?</p>
  </blockquote>


\end{minted}
\pagebreak

\section*{summarizer/condense.rb}
\begin{minted}[linenos=true,breaklines=true]{Ruby}
# condense.rb
#
# This module implements a means of merging strings that are similar. Making
# use of the differ gem, this can take a pair of strings
# e.g. "Abortion is not a right" and "Abortion is a right" and output:
# "Abortion is {not} a right".

module Condense
  def self.condense_group(group)
    return group if group.size < 2
    new_group = []
    merged = []
    pairs = group.uniq.combination(2).sort_by { |p1, p2| (p1.length - p2.length).abs }
    pairs.each do |s1, s2|
      originals = [s1, s2]
      next unless ((merged + new_group) & originals).empty?
      s1 = s1.downcase.gsub(/[^\s\w']/, "").strip
      s2 = s2.downcase.gsub(/[^\s\w']/, "").strip
      res = Differ.diff_by_word(s1, s2).to_s.gsub('"', "").gsub(" >> ", "|")
      change_count = res.scan(/\-|/).size
      group_count = res.scan(/\{[\w\|\s'\-]+\}/).size
      shared = (s1.split(/\s/) & s2.split(/\s/))
      if shared.empty? || group_count > 2 || change_count > 1 || change_count + group_count > 2
        new_group += originals
      else
        merged += originals
        new_group << merge_diff_groups(res)
      end
    end
    new_group += group - ((new_group + merged))
    (new_group - merged).uniq
      .map { |s| present_matched_string(s) }
      .sort_by { |s| s.index("{") || 1000 }
  end

  def self.merge_diff_groups(string)
    match = string.scan(/\{.+\} \{.+\}/).first
    return format_match_string(string) if match.nil?
    g1, g2 = match.split("} {").map { |e| e.gsub(/\{|\}/, "").split("|") }
    replacement = g1.zip(g2).map { |g| g.join(" ") }.join("|")
    format_match_string(string.gsub(match, " { #{replacement} } "))
  end

  def self.format_match_string(string)
    string.gsub(/\s*\}\s*/, "} ")
      .gsub(/\s*\{\s*/, " {")
      .gsub(/\s+/, " ")
      .gsub(/|\-/, "")
      .gsub(/\s?\|\s?/, "|")
      .gsub("{'t}", "{can't}") #https://en.wiktionary.org/wiki/Category:English_words_suffixed_with_-n%27t
      .gsub("{n't}", "{not}")
      .strip
  end

  def self.present_matched_string(string)
    string.gsub!(" '", "'")
    string[0].upcase + string[1..-1] + "."
  end
end


\end{minted}
\pagebreak

\section*{summarizer/counters.rb}
\begin{minted}[linenos=true,breaklines=true]{Ruby}
# counters.rb
#
# This file implements the search process for both counter and negated points.

module Counters
  PATTERN = /(\W|^)(n't|no|not|none|no one|nobody|nothing|neither|nowhere|never)(\W|$)/i
  def self.counter_points(points)
    antonyms = JSON.parse(File.open("antonyms.json").read)
    groups = points.group_by { |p| p["Components"] }

    counters = counters_for_points(points, groups, antonyms)

    counters.each do |k, v|
      v.each do |counter|
        counters.delete(counter) # remove mirrors
      end
    end

    counters.reject! { |k, _| k.size == 2 && k.join(" ").match(/(go|come)\.verb(\s|$)/) }

    counters.map {|k, v| v.map { |c| [k, c] }}
      .flatten(1)
      .sort_by { |k, v| groups[k].size + groups[v].size }
      .reverse
  end

  def self.negated_points(points)
    plain, neg = points.group_by { |p| p["String"].downcase.match(PATTERN).nil? }
      .map(&:last)
      .map { |g| Curator.reparse_points(Curator.select_best(g, true)) }

    return [] unless plain && neg
    neg.select! {|p| p["Relations"].include? "neg" }
    return [] if neg.empty?

    plain, neg = [plain, neg].map { |g| g.uniq { |p| Curator.clean_string(p["String"]) }  }

    counter_points = []
    plain.product(neg).each do |p1, p2|
      s1 = p1["String"].downcase.gsub(/[^\s\w']/, "").strip
      s2 = p2["String"].downcase.gsub(/[^\s\w']/, "").strip
      next if Levenshtein.distance(s1, s2) > 20
      res = Differ.diff_by_word(s1, s2).to_s.gsub('"', "").gsub(" >> ", "|")
      res = res.gsub("{+s}", "").gsub(/\|s\}/, "}").gsub(/\-|/, "")
      next unless res.include?("{")
      next if res.match(/\}\w\{/) || res.match(/\w(\{|\})\w/) || res.match(/\w(\{|\})\w/)
      next if res.chars.count("|") > 1
      next if (s1.split(" ").size - s2.split(" ").size).abs > 6
      next if (s1.split(" ") & s2.split(" ")).size < (s1.split(" ") + s2.split(" ")).size.to_f / 4
      next if res.scan(/\{[^\}]+\}/).size > 2
      next if res.scan(/\{[^\}]+\}/).join(" ").gsub(/[^\w\s]+/, " ").scan(/\w+ /).size > res.gsub(/[^\w\s]/, " ").scan(/\w+ /).size * 0.75
      next if res.scan(/\{[^\}]+\}/).map { |s| s.scan(/\w+ /).size }.max > 5
      counter_points << [res, p1, p2]
    end

    counter_point_groups = []
    counter_points.sort_by { |cp| cp.first.length  }.reverse.each do |cp|
      allocated = false
      counter_point_groups.each_with_index do |g, i|
        g.each do |s|
          next if Levenshtein.distance(cp.first, s.first) > 20
          allocated = true
          counter_point_groups[i] << cp
          break
        end
        break if allocated
      end
      counter_point_groups << [cp] unless allocated
    end
    return counter_point_groups
  end

  def self.replacement_options(point, antonyms)
    [].tap do |options|
      point["Components"].map { |c| c.split(".").first }.each do |w|
        options << (antonyms[w] || [w])
      end
    end
  end

  def self.counters_for_points(points, groups, antonyms)
    {}.tap do |counter_points|
      points.uniq { |p| p["Components"] }.each do |p|
        options = replacement_options(p, antonyms)

        next if options.flatten == p["Components"].map { |c| c.split(".").first }

        permutations = []
        options.each_with_index do |set, index|
          set.each do |replacement|
            new = p["Components"].dup
            new[index] = "#{replacement}.#{p["Components"][index].split(".").last}"
            next if new == p["Components"]
            permutations << new
          end
        end

        counter_points[p["Components"]] = permutations.select { |p| groups[p] }
      end
    end.reject! { |_, v| v.empty? }
  end
end


\end{minted}
\pagebreak

\section*{summarizer/curator.rb}
\begin{minted}[linenos=true,breaklines=true]{Ruby}
# curator.rb
#
# This implements the selection of the best extract for a list of points.
# This was tested in Study 2. select_best, given a list of points, will return
# the highest scoring one from the bigram model.

require 'net/http'

module Curator
  class CoreNLPClient
    def initialize(host)
      params = URI.encode('properties={"annotators": "lemma,tokenize,ssplit,depparse"}')

      @uri = URI(host + "/?" + params)
      @http_client = Net::HTTP.new(@uri.host, @uri.port)
    end

    def request_parse(text)
      req = Net::HTTP::Post.new(@uri)
      req.body = text
      JSON.parse(@http_client.request(req).body)['sentences'].map do |sentence|
        [sentence['tokens'], sentence['collapsed-ccprocessed-dependencies']]
      end
    end
  end

  def self.select_best(points, return_group=false)
    #original_points = points.dup
    points = reparse_points(points.uniq { |p| p["String"] })
    #if return_group
      #return points
    #else
      #return points.sample
    #end
    points = permitted(points)
    if points.empty?
      return_group ? (return []) : (return nil)
    end

    group_bigrams = []
    points.each do |p|
      group_bigrams += bigrams(p["String"])
    end
    group_bigrams = Utils.sorted_dup_hash(group_bigrams)
    points = points.sort_by do |p|
      clean = p["String"].downcase.gsub(/[^\s\w']/, "").gsub(/\s+/, " ").strip
      score = 0
      group_bigrams.each do |k, v|
        score += v if clean.include? k
      end
      p["Score"] = score.to_f / clean.split(" ").size
    end
    return points if return_group
    return points.last
  end

  def self.bigrams(string)
    words = string.downcase.gsub(/[^\s\w']/, "").split(/\s+/)
    words.each_with_index.to_a[0..-2].map do |e, i|
      [e, words[i+1]].join(" ")
    end
  end

  def self.reparse_points(points)
    cnc = CoreNLPClient.new("http://corenlp_server:9000")
    points.map do |point|
      parse = cnc.request_parse(clean_string(point["String"])).first
      point["Relations"], point["Lemmas"] = relations_and_lemmas_from_parse(parse)
      point
    end
  end

  def self.permitted(points)
    points.reject do |point|
      !point["String"].length.between?(15, 100) ||
      #point["String"].match(/^(if|and|but|or|just|after|before|their|his|her|why)/i) ||
      point["String"].downcase.match(/^\W*(who|what|when|where|why|which|how|whether)/) ||
      point["String"].include?("?") ||
      point["String"].match(/is\W+$/) ||
      point["String"].match(/this|And/) ||
      point["String"].match(/^\w+ ?, /) ||
      point["String"].match(/(\s*,){2}/) ||
      point["String"].match(/([A-Z]+ ){2,}/) ||
      point["String"].scan(/[a-z]+[0-9A-Z]/).any? ||
      point["String"].scan(/[LR]RB|[LR]SB/).size.odd? ||
      point["Relations"].count { |r| r.match(/dep|acl|csubj|ccomp/) } > 0 ||
      point["Relations"].count { |r| r.match(/advcl/) } > 1 ||
      point["Relations"].count { |r| r.match(/conj|nmod/) } > 2 ||
      point["Relations"].join(" ").match(/amod (punct)?$/) ||
      (!point["Lemmas"].join("").match(/VB/) && point["String"].length <= 20) ||
      contains_case_change(point) ||
      contains_bad_it(point) ||
      bad_repeated_word(point) ||
      boring_words(point, 2)
    end
  end

  def self.clean_string(string, question=false)
    string = string.strip
      .gsub(/^[^\w\{]+/, "")
      .gsub(/ ?- ?\.?$/, "")
      .gsub(/[\s;\.,]+$/, "")
      .gsub(/\s+(n?[,'\.])/, '')
      .gsub("-LRB-", "(").gsub("-RRB-", ")")
      .gsub("( ", "(").gsub(" )", ")")
      .gsub("-LSB-", "[").gsub("-RSB-", "]")
      .gsub("[ ", "[").gsub(" ]", "]").strip
      .gsub("` ", "'")
      .gsub(" '", "'")
      .gsub(/'([^t])/, '')
      .gsub(/([^\w\}]+\w{0,1})$/, "")
      .gsub("does not", "doesn't").gsub("can not", "can't").gsub("do not", "don't")
      .gsub(" i ", " I ")
      .gsub(/[^A-Za-z\)\}\]]+$/, "")
    unless question
      string.gsub!(/^(therefor|therefore|then|than|so|to|when|what|that|even if|if|of|even|about|because)\s/i, "")
    end
    string = "#{string[0].upcase}#{string[1..-1]}"
    string.gsub!(/[;.]+/, "")
    string.gsub!(/ [:\-]$/, "")
    string.gsub!(" ,", "")
    string = string.strip
    if question
      string += "?" if string.strip[-1] != "?"
    else
      string += "." if string.strip[-1] != "."
    end
    string
  end

  def self.relations_and_lemmas_from_parse(parse)
    relations = parse.last.group_by { |r| r["dependent"] }
      .map { |k, v| [k-1, v.map { |r| r["dep"] }] }
      .sort_by { |index, _| index }
      .map(&:last).map { |r| r.join("|") }
    return relations, parse.first.map { |t| t["lemma"] + ":" + t["originalText"] + ":" + t["pos"] }
  end

  def self.contains_case_change(point)
    return false unless point["String"].match(/\s[A-Z]/)
    return true if point["String"].match(/\?|\.[A-Z]/)
    point["Lemmas"].count do |e|
      next unless e.match(/\w/) && point["Lemmas"].index(e) > 0
      e = e.split(":")
      e[1].match(/^[A-Z]/) && !e[2].match(/^NN|^PRP/) && e[1].upcase != e[1]
    end > 0
  end

  def self.contains_bad_it(point)
    return false unless point["Lemmas"].map { |l| l.split(":").first}.include? "it"
    clean_text = point["String"].downcase
    return false if clean_text.include? "act on it"
    !clean_text.match(/(and|but|whether)\sit\s/)
  end

  def self.boring_words(point, count)
    point["Lemmas"].count do |e|
      next unless e.match(/\w/)
      e = e.split(":")
      e[2].match(/^NN|^JJ|RB/)
    end < count
  end

  def self.bad_repeated_word(point)
    point["Lemmas"].reject { |l| l.match(/:{2,}/) }
      .map { |l| l.split(":").first }
      .inject { |last, e| break if last == e && e.match(/\w/) && !e.match(/be|have/); last = e }.nil?
  end
end


\end{minted}
\pagebreak

\section*{summarizer/paragraphizer.rb}
\begin{minted}[linenos=true,breaklines=true]{Ruby}
# paragraphizer.rb
#
# This file is was used in the generation of the Plain summary style for the
# evaluation. It is not used in the current summarizer implementation to keep
# the basic analysis process as simple as possible.

module Paragraphizer
  def self.generate_paragraph(data)
    paragraph = ""

    data[:counter_points].each_with_index do |points, index|
      p, c = points
      paragraph += Presenter.clean(p["String"]) + "\n"
    end

    data[:negated_points].each_with_index do |point, index|
      paragraph += Presenter.clean(point[1]["String"]) + "\n"
    end

    data[:related_points].each_with_index do |points, index|
      p, r = points
      paragraph += Presenter.clean(p["String"]) + "\n"
      paragraph += Presenter.clean(r["String"]) + "\n"
    end

    data[:common_points].each_with_index do |p, index|
      paragraph += Presenter.clean(p["String"]) + "\n"
    end

    data[:longer_points].each do |p|
      paragraph += Presenter.clean(p["String"]) + "\n"
    end

    data[:multiple_topic_points].each_with_index do |p, index|
      paragraph += Presenter.clean(p["String"]) + "\n"
    end

    data[:commonly_discussed_topic_points].each_with_index do |topic, index|
      topic, points = topic
      points.each_with_index do |p, index|
        paragraph += Presenter.clean(p["String"]) + "\n"
      end
    end

    data[:question_points].each_with_index do |p, index|
      paragraph += Presenter.clean(p["String"], true) + "\n"
    end

    return paragraph
  end
end


\end{minted}
\pagebreak

\section*{summarizer/presenter.rb}
\begin{minted}[linenos=true,breaklines=true]{Ruby}
# presenter.rb
#
# This module implements the clean and format methods. Given a string such
# as “, abortion is a right - ” it will output “Abortion is a right.”.
# Every extract is cleaned using the methods defined here before being
# included as part of a summary.

module Presenter
  def self.format(string, topics, question=false)
    highlight(clean(string, question), topics)
  end

  def self.clean(string, question=false)
    string = string.strip
      .gsub(/^[^\w\{]+/, "")
      .gsub(/ ?- ?\.?$/, "")
      .gsub(/[\s;\.,]+$/, "")
      .gsub(/\s+(n?[,'\.])/, '')
      .gsub("-LRB-", "(").gsub("-RRB-", ")")
      .gsub("( ", "(").gsub(" )", ")")
      .gsub("-LSB-", "[").gsub("-RSB-", "]")
      .gsub("[ ", "[").gsub(" ]", "]").strip
      .gsub("` ", "'")
      .gsub(/'([^tmlsrvd])/, '')
      .gsub(/([^\w\}]+\w{0,1})$/, "")
      .gsub("does not", "doesn't").gsub("can not", "can't").gsub("do not", "don't")
      .gsub(" i ", " I ")
      .gsub(/[^A-Za-z\)\}\]]+$/, "")
    unless question
      string.gsub!(/^(therefor|therefore|then|than|so|to|when|what|that|even if|if|of|even|about|because)\s/i, "")
    end
    string = "#{string[0].upcase}#{string[1..-1]}" rescue binding.pry
    string.gsub!(/[;.]+/, "")
    string.gsub!(/ [:\-]$/, "")
    string.gsub!(" ,", "")
    string = string.strip
    if question
      string += "?" if string.strip[-1] != "?"
    else
      string += "." if string.strip[-1] != "."
    end
    string
  end

  def self.highlight(string, topics)
    string = string.gsub("|", "<strong> or </strong>")
      .gsub("{", "<strong> (</strong>")
      .gsub("}", "<strong>) </strong>")
    topic_regex = topics.sort_by(&:length).reverse.join("|")
    string.scan(/#{topic_regex}/i).each { |match| string.gsub!(/(^|\W)#{match}(\W)/, '' + "<code>#{match}</code>" + '') }
    string.strip
  end
end


\end{minted}
\pagebreak

\section*{summarizer/related.rb}
\begin{minted}[linenos=true,breaklines=true]{Ruby}
# related.rb
#
# This implements the search for related points. Using the post source of
# points, they can be grouped. These commonly occurring pairs of patterns make
# up the list of related point pairs.

module Related
  def self.related_points(points)
    posts = points.group_by { |p| p["Post"] }

    ptoi = {}
    itop = {}
    points.each_with_index { |p, i| ptoi[p["Components"]] = i; itop[i] = p }

    posts = posts.map { |k, v| v.map { |p| ptoi[p["Components"]] }.uniq }

    related = Hash.new(0)
    posts.each do |p|
      p.combination(2).to_a.each do |c|
        related[c] += 1
      end
    end

    related = related.reject { |k,v| v < 3 }
    related = related.sort_by { |k, v| v }.reverse

    related.map do |points, count|
      points.map! { |p| itop[p] }
      if (points.first["Components"] & points.last["Components"]).size > 1
        next
      end
      [points.map { |p| p["Components"] }, count]
    end.compact.map(&:first)
  end
end


\end{minted}
\pagebreak

\section*{summarizer/summarizer.rb}
\begin{minted}[linenos=true,breaklines=true]{Ruby}
# summarizer.rb
#
# This script brings all the other modules in this service together and uses
# them to analyze points and produce a summary from them.

require 'json'
require 'erb'

require 'pry'
require 'differ'
require 'levenshtein'

require_relative 'summary'
require_relative 'utils'
require_relative 'related'
require_relative 'counters'
require_relative 'curator'
require_relative 'condense'
require_relative 'presenter'
require_relative 'paragraphizer'

# compute the title from the input file name
title = ARGV[0].scan(/(\w+)_p/).flatten.first
  .gsub(/(.)([A-Z])/,' ').split(" ").map(&:capitalize).join(" ")

out_file_name = title.split(/\W+/).join("_").downcase
in_file_name = title.split(" ").map(&:capitalize).join.tap { |e| e[0] = e[0].downcase }

# read, parse and use points in the generation of a summary
lines = File.open(ARGV[0]).readlines
topics = lines.first.split(",").map(&:strip)
points = lines[2..-1].map { |l| JSON.parse(l) }
summary = Summary.new(title, points, topics, 3)
summary.build

@title = title
@summary = summary

# use the template to generate the html output
erb = ERB.new(File.open("template_formatted.html.erb").read, 0, '>')
html = erb.result binding
File.open(out_file_name + "_formatted.html", "w") { |file| file.write(html) }


\end{minted}
\pagebreak

\section*{summarizer/summary.rb}
\begin{minted}[linenos=true,breaklines=true]{Ruby}
# summary.rb
#
# This file implements the class for a summary object. Summary instances are
# built up in the build method

class Summary
  attr_accessor :topics, :counter_points, :related_points, :negated_points,
    :common_points, :longer_points, :commonly_discussed_topic_points,
    :multiple_topic_points, :question_points, :point_count

  # a title, list of points, topics, and desired point count are inputs
  def initialize(title, points, topics, point_count)
    @title, @points, @topics, @point_count = title, points, topics, point_count
    @groups = Hash[*points.group_by { |p| p["Components"] }.sort_by { |_, v| v.size }.reverse.flatten(1)]
    @used_points = []
  end

  # each section of the summary is built in turn, this takes some time
  def build
    @counter_points = generate_counter_points; print "."
    @related_points = generate_related_points; print "."
    @negated_points = generate_negated_points; print "."
    @common_points = generate_common_points; print "."
    @longer_points = generate_longer_points; print "."
    @commonly_discussed_topic_points = generate_commonly_discussed_topic_points; print "."
    @multiple_topic_points = generate_multiple_topic_points; print "."
    @question_points = generate_question_points; print ".\n"
  end

  def generate_counter_points
    count = 0
    Counters.counter_points(available_points).map do |point, counter|
      next if count >= @point_count
      point = Curator.select_best(@groups[point])
      counter = Curator.select_best(@groups[counter])
      next if [point, counter].map(&:nil?).any?
      count += 1
      use_point(point); use_point(counter)
      [point, counter]
    end.compact
  end

  def generate_related_points
    count = 0
    used_related_points = []
    Related.related_points(available_points).map do |point, related|
      next if count >= @point_count
      point = Curator.select_best(@groups[point])
      related = Curator.select_best(@groups[related])
      next if [point, related].map(&:nil?).any?
      next unless (used_related_points & [point, related]).empty?
      count += 1
      used_related_points << point << related
      use_point(point); use_point(related)
      [point, related]
    end.compact
  end

  def generate_negated_points
    count = 0
    points = @groups.take(100).map do |_, group|
      next if count >= @point_count * 2
      counter_point_groups = Counters.negated_points(available_points(group))
      next if counter_point_groups.empty?
      best = counter_point_groups.map { |g| g.min_by { |r, _, _| r.length } }.min_by { |r, _, _| r.scan(/\||\{|\}/).size }
      next unless best
      count += 1
      use_point(best[1]); use_point(best[2])
      best[0] = Curator.clean_string(Condense.format_match_string(Condense.merge_diff_groups(best.first)))
      best
    end.compact.sort_by! { |best, _, _| best.scan(/\{|\}|\|/).size }
    if points.size > @point_count
      points.reverse.take(points.size - @point_count).each do |_, p1, p2|
        return_point(p1)
        return_point(p2)
      end
    end
    points.take(@point_count)
  end

  def generate_common_points
    count = 0
    @groups.map do |k, g|
      next if count >= @point_count
      best = Curator.select_best(available_points(g))
      next if best.nil?
      count += 1
      use_point(best)
    end.compact
  end

  def generate_longer_points
    used_topics = []
    points = []
    @groups.reject { |k, v| k.size < 4 || v.uniq { |p| p["String"] }.size < 2 }.sort_by { |_, v| v.size }.reverse.map do |k, g|
      g.uniq! { |p| p["String"] }
      best = Curator.select_best(available_points(g))
      next unless best
      topics = @topics.select { |t| best["String"].downcase.include?(t) || best["Lemmas"].select { |l| l.match(/[a-z]/) }.map { |l| l.split(":").first }.include?(t) }
      next if (used_topics & topics).size > 1
      next if points.map { |p| p["String"] }.include? best["String"]
      used_topics += topics
      points << best
    end
    points.sort_by do |point|
      @topics.select do |t|
        point["String"].downcase.include?(t) ||
        point["Lemmas"].select { |l| l.match(/[a-z]/) }.map { |l| l.split(":").first }.include?(t)
      end.size
    end.reverse.take(@point_count)
  end

  def generate_commonly_discussed_topic_points
    count = 0
    top_topics.map do |t|
      next if count >= @point_count
      topic_points = []
      used_components = []
      @groups.select { |k, _| k.join(" ").include? " #{t}." }.each do |k, group|
        best = Curator.select_best(available_points(group))
        next unless best
        next if (best["Components"] - used_components).size < 1
        size = Condense.condense_group(topic_points.compact.map { |p| Presenter.clean(p["String"]) }).size
        next_size = Condense.condense_group(topic_points.compact.map { |p| Presenter.clean(p["String"]) } + [Presenter.clean(best["String"])]).size
        next if size >= @point_count || next_size == size
        topic_points << best
        used_components += best["Components"]
      end
      count += 1
      [t, topic_points.compact]
    end.compact
  end

  def generate_multiple_topic_points
    topic_points = @points.sort_by { |p| @topics.count { |t| p["String"].downcase.include? t } }.reverse.take(100)
    topic_points.uniq! { |p| p["String"] }
    selected_points = []
    for i in 0..15
      selected_points << topic_points.delete(Curator.select_best(topic_points))
      break if selected_points.compact.size >= @point_count
    end
    selected_points.compact
  end

  def generate_question_points
    question_groups = @points.select { |p| p["String"].strip[-1] == "?" }
          .uniq {|p| p["String"] }
          .group_by { |p| p["Components"] }
          .sort_by { |k, v| v.size }
          .select { |k, v| v.size > 2 }
    count = 0
    question_groups.map do |_, group|
      next if count >= @point_count
      count += 1
      group.sort_by { |p| p["String"].length }.first
    end.compact
  end

  def metadata
    {
      points: @points.size,
      groups: @groups.size,
      posts: @points.group_by { |p| p["Post"] }.size,
    }
  end

  def to_h
    {
      counter_points: @counter_points,
      related_points: @related_points,
      negated_points: @negated_points,
      common_points: @common_points,
      longer_points: @longer_points,
      commonly_discussed_topic_points: @commonly_discussed_topic_points,
      multiple_topic_points: @multiple_topic_points,
      question_points: @question_points,
    }
  end

  private

  # these private methods implement the point `spending' concept used in avoiding reuse

  def available_points(points=@points)
    points.select { |p| (@used_points & point_id_strings(p)).empty? }
  end

  def use_point(point)
    @used_points += point_id_strings(point)
    point
  end

  def return_point(point)
    @used_points -= point_id_strings(point)
    point
  end

  def point_id_strings(point)
    if point["Lemmas"]
      lemma_string = point["Lemmas"].select { |l| l.match(/[a-z]/) }.reject { |l| l.match(/IN|DT|MD/) }.map { |l| l.split(":").first }.join(" ").downcase
    end
    clean_string = point["String"].gsub(/[^\w\s]+/, "").gsub(/\s+/, " ").strip.downcase.gsub(/\s(at|from)\s/, " ")
    formatted_string = Presenter.clean(point["String"])
    [
      lemma_string,
      clean_string,
      clean_string.gsub(" ", ""),
      formatted_string,
      formatted_string.gsub(" ", ""),
      point["Components"].select { |c| c.match(/\.\w*(verb|subj|obj)/) }.map { |c| c.split(".").first }.join(" ")
    ].compact
  end

  def top_topics
    Utils.sorted_dup_hash(@groups.keys.flatten.map(&:downcase))
      .keys.select { |e|
        e.match(/nsubj|dobj/) &&
        !e.match(/person|they|\.verb|\.prep|it\.|what\.|that\.|one\./)
      }.map { |e| e.split(".").first }.uniq
  end
end


\end{minted}
\pagebreak

\section*{summarizer/template\_formatted.html.erb}
\begin{minted}[linenos=true,breaklines=true]{XML+Ruby}
<p>People <strong>disagree</strong> on these points:</p>
<% @summary.counter_points.each do |point, counter| %>
  <% point = [point["String"], counter["String"]] %>
  <% if point.size == 2 %>
    <blockquote>
      <p><%= Presenter.format(point.first, @summary.topics) %></p>
    </blockquote>
  <% else %>
    <blockquote><p><%= Presenter.format(point.first, @summary.topics) %></p></blockquote>
  <% end %>
<% end %>

<% @summary.negated_points.each do |_, first, _| %>
  <blockquote>
    <p><%= Presenter.format(first["String"], @summary.topics) %></p>
  </blockquote>
<% end %>

<p><strong>Commonly occurring points</strong> made in the discussion were:</p>
<% @summary.common_points.each do |point| %>
  <blockquote>
    <p><%= Presenter.format(point["String"], @summary.topics) %></p>
  </blockquote>
<% end %>

<p>Users that talk about <strong><em>X</em> often talk about <em>Y</em></strong>.</p>
<% @summary.related_points.each do |point, related| %>
  <blockquote>
    <p>
    <strong>(<em>X</em>)</strong>
      <%= Presenter.format(point["String"], @summary.topics) %>
      <strong>(<em>Y</em>)</strong>
      <%= Presenter.format(related["String"], @summary.topics) %>
    </p>
  </blockquote>
<% end %>

<p>Common points made in the discussion that <strong>link terms</strong> were:</p>
<% @summary.longer_points.each do |point| %>
  <blockquote>
    <p><%= Presenter.format(point["String"], @summary.topics) %></p>
  </blockquote>
<% end %>

<p>Points for <strong>commonly discussed topics</strong>:</p>
<ul>
  <% @summary.commonly_discussed_topic_points.each do |topic, points| %>
    <li>
      <p><%= topic.capitalize %>
      <% points.map { |p| Presenter.clean(p["String"]) }.take(@summary.point_count).each do |point| %>
        <blockquote>
          <p><%= Presenter.format(point, @summary.topics) %></p>
        </blockquote>
      <% end%>
    </li>
  <% end %>
</ul>

<p>Points about <strong>multiple topics</strong>:</p>
<% @summary.multiple_topic_points.each do |point| %>
  <blockquote>
    <p><%= Presenter.format(point["String"], @summary.topics) %></p>
  </blockquote>
<% end %>

<p>People ask <strong>questions</strong> like:</p>
<% if @summary.question_points.empty? %>
  <blockquote>
    <p>There were no repeated questions</p>
  </blockquote>
<% end %>
<% @summary.question_points.each do |point| %>
  <blockquote>
    <p><%= Presenter.format(point["String"], @summary.topics, true) %></p>
  </blockquote>
<% end %>


\end{minted}
\pagebreak

\section*{summarizer/utils.rb}
\begin{minted}[linenos=true,breaklines=true]{Ruby}
# utils.rb
#
# This file contains a utility function for collecting a key value
# representation of the duplicates in a list. This is used in throughout the
# service.

module Utils
  def self.sorted_dup_hash(array)
    Hash[*array.inject(Hash.new(0)) { |h,e| h[e] += 1; h }.
      select { |k,v| v > 1 }.inject({}) { |r, e| r[e.first] = e.last; r }.
      sort_by {|_,v| v}.reverse.flatten]
  end
end


\end{minted}
\pagebreak

\section*{topic\_api/app.rb}
\begin{minted}[linenos=true,breaklines=true]{Ruby}
# app.rb
#
# This file implements the topic api service. It exposes a route that can
# accept requests containing a body of text. From this is runs the LDA
# analysis and returns the topic words as a response.

require 'sinatra'
require 'lda-ruby'
require 'json'

set :port, ENV['PORT']
set :bind, '0.0.0.0'

class TopicAnalyzer
  def initialize(text, topic_count)
    corpus = Lda::Corpus.new
    corpus.add_document(Lda::TextDocument.new(corpus, text))
    @lda = Lda::Lda.new(corpus)
    @lda.num_topics = topic_count
    @lda.em('random')
  end

  def top_words(count)
    @lda.top_words(count).values
  end
end

blacklist = ["-", "http", "https", "feel", "make", "right", "wrong", "things"]

get "/" do
  "Post to this route with a JSON payload. (text, topic_count, top_word_count)"
end

post '/' do
  payload = JSON.parse(request.body.read)
  text = payload["text"].encode("UTF-8", :invalid=>:replace, :replace=>"?")

  ta = TopicAnalyzer.new(text, payload["topic_count"].to_i)
  groups = ta.top_words(payload["top_word_count"].to_i)

  groups.map! { |g| g - blacklist }

  {
    topics: groups.flatten.uniq.sort,
    groups: groups
  }.to_json
end


\end{minted}
\pagebreak

\section*{curator/main.go}
\begin{minted}[linenos=true,breaklines=true]{Go}
// main.go
//
// This implements the points curation task

package main

import (
	"encoding/json"
	"fmt"
	"io/ioutil"
	"os"
	"regexp"
	"strings"
)

// this represents points as they are saved by the points extraction tool
type RawPoint struct {
	String             string
	Pattern            string
	Stance             string
	OriginalStanceText string
	OriginalTopic      string
	Post               string
	Content            string
}

// this process upgrades points, adding attributes to aid later analysis
type Point struct {
	String             string
	Verb               string
	Stance             string
	OriginalStanceText string
	OriginalTopic      string
	Post               string
	Content            string `json:"-"`
	Components         []string
	Words              []string
	Relations          []string
}

func (p *Point) print() {
	fmt.Printf("(%v) : %v\n", strings.Join(p.Components, " "), p.String)
}

func (p *Point) componentString() string {
	return strings.Join(p.Components, " ")
}

func (p *Point) matches(p2 Point) bool {
	if p.Verb != p2.Verb {
		return false
	}
	return p.matchesWords(p2) || p.componentString() == p2.componentString()
}

func (p *Point) matchesWords(p2 Point) bool {
	return compare(p.Words, p2.Words) && compare(p2.Words, p.Words)
}

func compare(X, Y []string) bool {
	m := make(map[string]int)

	for _, y := range Y {
		m[y]++
	}

	var ret []string
	for _, x := range X {
		if m[x] > 0 {
			m[x]--
			continue
		}
		ret = append(ret, x)
	}

	return len(ret) == 0
}

func containsStr(s []string, e string) bool {
	for _, a := range s {
		if a == e {
			return true
		}
	}
	return false
}
func containsInt(s []int, e int) bool {
	for _, a := range s {
		if a == e {
			return true
		}
	}
	return false
}

func removeDuplicates(elements []int) []int {
	encountered := map[int]bool{}
	result := []int{}

	for v := range elements {
		if encountered[elements[v]] == true {
		} else {
			encountered[elements[v]] = true
			result = append(result, elements[v])
		}
	}
	return result
}

func pairs(arr []int) [][]int {
	var pairs [][]int
	for i := 0; i < len(arr); i++ {
		for j := i + 1; j < len(arr); j++ {
			pairs = append(pairs, []int{arr[i], arr[j]})
		}
	}
	return pairs
}

// this function uses the point's existing attributes to generate some new ones
// these new attributes make the points easier to process
// for example, point patterns are split into separate components
func upgradePoint(p RawPoint) Point {
	pattern := regexp.MustCompile("cop|pass").ReplaceAllString(p.Pattern, "")
	components := strings.Split(pattern, " ")
	var verb string
	var words []string
	var relations []string
	for _, v := range components {
		parts := strings.Split(v, ".")
		if parts[1] == "verb" {
			verb = parts[0]
		}
		words = append(words, parts[0])
		relations = append(relations, parts[1])
	}
	return Point{
		String:             p.String,
		Stance:             p.Stance,
		OriginalStanceText: p.OriginalStanceText,
		OriginalTopic:      p.OriginalTopic,
		Post:               p.Post,
		Content:            p.Content,
		Verb:               verb,
		Components:         components,
		Relations:          relations,
		Words:              words,
	}
}

type ByLen [][]Point

func (a ByLen) Len() int           { return len(a) }
func (a ByLen) Swap(i, j int)      { a[i], a[j] = a[j], a[i] }
func (a ByLen) Less(i, j int) bool { return len(a[i]) > len(a[j]) }

// main task function
func main() {
	// read in the list of points
	b, err := ioutil.ReadFile(os.Args[1])
	if err != nil {
		panic(err)
	}

	// parse points into rawpoints and then upgrade them
	points := []Point{}
	contents := strings.Split(string(b), "\n")
	for i, v := range contents {
		if i == 0 {
			fmt.Println(v)
			continue
		} else if len(v) < 4 {
			continue
		} else if v[0] != '{' {
			continue
		}
		var rawPoint = RawPoint{}
		json.Unmarshal([]byte(v[0:len(v)-1]), &rawPoint)
		points = append(points, upgradePoint(rawPoint))
	}

	// these are the blacklists that are applied to points
	bannedList := strings.Split("it.nsubj that.nsubj this.nsubj which.nsubj what.nsubj", " ")
	bannedPersonList := strings.Split("object continue come go sit open close begin end believe happen leave understand realize debate speak show stand call refer believe lose change care hear write disagree read tell start talk explain come live take support guess feel follow make go get move agree find fail think wonder feel ask argue try", " ")
	for i, v := range bannedPersonList {
		bannedPersonList[i] = fmt.Sprintf("%v.verb", v)
	}
	bannedComponentList := []string{
		"PERSON.nsubj be.verb correct.dobj",
		"PERSON.nsubj be.verb able.dobj",
		"PERSON.nsubj be.verb good.dobj",
		"PERSON.nsubj be.verb likely.dobj",
		"PERSON.nsubj be.verb sorry.dobj",
		"PERSON.nsubj be.verb say.dobj",
		"PERSON.nsubj be.verb aware.dobj",
		"PERSON.nsubj be.verb one.dobj",
		"PERSON.nsubj be.verb sure.dobj",
		"PERSON.nsubj be.verb wrong.dobj",
		"PERSON.nsubj be.verb glad.dobj",
		"PERSON.nsubj be.verb here.dobj",
		"PERSON.nsubj be.verb willing.dobj",
		"PERSON.nsubj be.verb right.dobj",
		"PERSON.nsubj be.verb true.dobj",
		"PERSON.nsubj be.verb false.dobj",
		"PERSON.nsubj be.verb favor.dobj",
		"PERSON.nsubj be.verb interested.dobj",
		"PERSON.nsubj want.verb have.xcomp",
		"PERSON.nsubj want.verb what.dobj",
		"PERSON.nsubj want.verb what.dobj do.xcomp",
		"PERSON.nsubj say.verb what.dobj",
		"PERSON.nsubj mean.verb what.dobj",
		"PERSON.nsubj know.verb what.dobj",
		"PERSON.nsubj believe.verb what.dobj",
		"PERSON.nsubj see.verb what.dobj",
		"PERSON.nsubj see.verb argument.dobj",
		"PERSON.nsubj have.verb problem.dobj",
		"PERSON.nsubj tell.verb they.dobj",
		"PERSON.nsubj think.verb what.dobj",
		"PERSON.nsubj argue.verb in.prep fact.dobj",
		"PERSON.nsubj argue.verb with.prep you.dobj",
		"debate.nsubj be.verb about.dobj",
		"question.nsubj be.verb",
		"make.verb claim.dobj",
		"ask.verb yourself.dobj",
		"thing.nsubj happen.verb",
		"something.nsubj happen.verb",
	}
	personList := strings.Split("I.nsubj i.nsubj who.nsubj we.nsubj I.nsubj you.nsubj they.nsubj he.nsubj she.nsubj person.nsubj people.nsubj", " ")

	originalSize := len(points)

	// This loop applies the blacklist rejections and nsubj amalgamations
	for i := 0; i < len(points); i++ {
		point := points[i]
		if containsStr(personList, point.Components[0]) {
			point.Components[0] = "PERSON.nsubj"
		}
		if containsStr(bannedList, point.Components[0]) {
			points = append(points[:i], points[i+1:]...)
			i--
		} else if len(point.Components) == 2 && point.Components[0] == "PERSON.nsubj" && containsStr(bannedPersonList, point.Components[1]) {
			points = append(points[:i], points[i+1:]...)
			i--
		} else if containsStr(bannedComponentList, strings.Join(point.Components, " ")) {
			points = append(points[:i], points[i+1:]...)
			i--
		} else if len(point.String) < 10 {
			points = append(points[:i], points[i+1:]...)
			i--
		}
	}
	fmt.Printf("%v of %v points disqualified\n", originalSize-len(points), originalSize)

	// this prints the selected / upgraded points to disk in a new points file
	for _, v := range points {
		b, err := json.Marshal(v)
		if err != nil {
			return
		}
		fmt.Println(string(b))
	}
}


\end{minted}
\pagebreak

\section*{evaluation/extract\_ranking.rb}
\begin{minted}[linenos=true,breaklines=true]{Ruby}
# extract_ranking.rb
#
# This is a script that contains the same bigram ranking code used in
# generating summaries. It can be used to get scores for a group of extracts.

require 'pry'

# extend array for basic stats
# Source http://stackoverflow.com/a/7749613/1510063
module Enumerable
  def sum
	self.inject(0){|accum, i| accum + i }
  end

  def mean
	self.sum/self.length.to_f
  end

  def sample_variance
	m = self.mean
	sum = self.inject(0){|accum, i| accum +(i-m)**2 }
	sum/(self.length - 1).to_f
  end

  def standard_deviation
	return Math.sqrt(self.sample_variance)
  end
end

def sorted_dup_hash(array)
  Hash[*array.inject(Hash.new(0)) { |h,e| h[e] += 1; h }.
    select { |k,v| v > 1 }.inject({}) { |r, e| r[e.first] = e.last; r }.
    sort_by {|_,v| v}.reverse.flatten]
end

# for a string, return the all the bigrams
def self.bigrams(string)
  words = string.downcase.gsub(/[^\s\w']/, "").split(/\s+/)
  words.each_with_index.to_a[0..-2].map do |e, i|
    [e, words[i+1]].join(" ")
  end
end

# this method implements the bigram model
def score_group(points)
  group_bigrams = []
  points.each do |p|
      group_bigrams += bigrams(p["String"])
  end
  group_bigrams = sorted_dup_hash(group_bigrams)
  points = points.sort_by do |p|
    clean = p["String"].downcase.gsub(/[^\s\w']/, "").gsub(/\s+/, " ").strip
    score = 0
    group_bigrams.each do |k, v|
      score += v if clean.include? k
    end
    p["Score"] = score.to_f / clean.split(" ").size
  end
end

# load the extracts and score them
bigram_scored_extracts = []
Dir.glob('extracts/*') do |p|
  File.open(p).read.split('------').each do |group|
    extracts = group.split("\n").reject { |l| l.length < 2 || l[0].match(/\W/) }
    bigram_scored_extracts += score_group(extracts.map { |e| { "String" => e } }).map { |e| [e["String"], e["Score"]] }
  end
end
bigram_scored_extracts = Hash[*bigram_scored_extracts.flatten]

# load the turker scored extracts
human_scored_extracts = Hash[*File.open('scored_extracts.txt').readlines.map { |x| [[x[x.index(",")+2..-2]], x[0..x.index(",")-1].to_f] }.flatten]

# print both scores for each extract
puts "bigram, turkers"
bigram_scored_extracts.each do |k, v|
  raise if human_scored_extracts[k].nil?
  puts [v.round(3), human_scored_extracts[k]].join(", ")
end
exit

machine_mean = bigram_scored_extracts.values.mean
machine_stdev = bigram_scored_extracts.values.standard_deviation

machine_for_each_human_score = Hash.new([])
File.open('scored_extracts_all.txt').readlines.map { |l| l.split("||") }.each do |e, scores|
  next unless e.strip.length > 0
  scores.strip.split(",").map(&:to_i).each do |s|
    machine_score = bigram_scored_extracts[e]
    raise if machine_score.nil?
    machine_for_each_human_score[s] += [((machine_score - machine_mean) / machine_stdev).round(2)]
  end
end

machine_for_each_human_score[1] += machine_for_each_human_score.delete(0)

machine_for_each_human_score.sort_by { |k, _| k }.each do |k, v|
  puts "r#{k} = c(#{v.join(",")})"
end


\end{minted}
\pagebreak

\section*{evaluation/process\_extract\_comparison.rb}
\begin{minted}[linenos=true,breaklines=true]{Ruby}
# process_extract_comparison.rb
#
# This is a general purpose script to process the results from Study 2.
# It prints the results in a variety of formats.

require 'pry'
require 'csv'

responses = []
Dir.glob('./extract_comparison/ext*') do |path|
  rows = []
  CSV.foreach(path) do |row|
    rows << row
  end
  headers, rows = rows[0], rows[1..-1]
  responses += rows.map { |r| Hash[*headers.zip(r).flatten] }
end

puts "Unique Workers: #{responses.map { |x| x["WorkerId"] }.uniq.size}"
answers = []

bigram_v_rand_answers = []
comments = []

responses.each_with_index do |r, i|
  r.select { |k, _| k.include? "Answer" }.sort_by { |_, v| v.length }.each do |q, a|
    topic, id =  q.gsub("Answer.", "").split("-")
    if id.match(/set[123]extract[0-9]+/)
      a = a.gsub(",", ";").gsub(/\s+/, " ").to_i
      answers << [topic + id, a, i]
    elsif a.match(/^\w+\-\w+$|^same$/)
      bigram_v_rand_answers << a
    else
      last_response = r.select { |k, v| k.include? id }.first.last
      comments << { topic: topic, comment: a, last_response: last_response }
    end
  end
end

topic_sets = []
Dir.glob('./extracts/ext*') do |path|
  contents = File.open(path).read
  sets = contents.split("------").map { |g| g.split("\n").reject { |l| l == "" } }.map do |g|
    extracts = g[1..-2]
    extracts[extracts.index(g[-1][2..-1])] = "(SELECTED) " + g[-1][2..-1]
    extracts
  end
  topic = path.match(/\/extracts_(\w+)\.txt/)[1]
  topic_sets << [topic, sets]
end

scored_sentences = []
topic_sets.each do |topic, sets|
  sets.each_with_index do |set, set_index|
    set.each_with_index do |extract, extract_index|
      key = "#{topic}set#{set_index + 1}extract#{extract_index + 1}"
      scores = answers.select { |k, _| k == key }.map(&:last)
      scores = answers.select { |k, _| k == key }.map { |_, v, _| v }
      score = (scores.reduce(:+).to_f / scores.size)
      scored_sentences << [extract, score]
    end
  end
end

less_than = 0
more_than = 0
answers.group_by(&:last).each do |k,v|
  average = v.map { |a, s, _| s }.reduce(:+).to_f / v.size
  topic = v.first.first.split("set").first
  topic_sets.select { |k, v| k == topic }.each do |k, sets|
    sets.each_with_index do |set, index|
      index += 1
      ans = Hash[*v.select { |k, _, _| k.include? "set#{index}" }.map { |k, v| [k.scan(/[0-9]+$/).first.to_i, v] }.flatten]
      selected_index = set.index(set.select { |v| v.include? "SELECTED" }.first) + 1
      score_for_selected = ans[selected_index]
      score_for_selected < average ? less_than += 1 : more_than += 1
    end
  end
end

puts "Better than avg: #{more_than.to_f/(more_than+less_than)}"

answers.map! { |x, y, _| [x, y] }

print "All Extracts: "
p mean = scored_sentences.map(&:last).reduce(:+) / scored_sentences.size

print "Selected Extracts: "
selected = scored_sentences.select { |s| s.first.include? "SELECTED" }
puts selected_mean = selected.map(&:last).reduce(:+) / selected.size

print "Other Extracts: "
rejected = scored_sentences - selected
puts rejected_mean = rejected.map(&:last).reduce(:+) / rejected.size

selected.sort_by(&:last).reverse.map { |e, s| puts "#{s.round(3)} - #{e}" }
rejected.sort_by(&:last).reverse.map { |e, s| puts "#{s.round(3)} - #{e}" }

puts "\nComments\n"
comments.each do |comment|
  puts [
    comment[:topic].upcase,
    "A was bigram, B was random",
    "rating given: " + comment[:last_response].gsub(/^layout/, "bigram_layout")
  ].join(", ")
  text = comment[:comment]
  text.gsub!(/(^|\W)A\W/, " [BIGRAM] ")
  text.gsub!(/(^|\W)B\W/, " [RANDOM] ")
  puts text.strip
  puts "-"*100
end

puts "\nRatings:"
["layout-much_better", "layout-better", "same", "random_layout-better", "random_layout-much_better"].each do |a|
  puts ["bigram_vs_random", "overall", a, bigram_v_rand_answers.count(a).to_i].join(",")
end


\end{minted}
\pagebreak

\section*{evaluation/process\_summary\_comparison.rb}
\begin{minted}[linenos=true,breaklines=true]{Ruby}
# process_summary_comparison.rb
#
# This is a general purpose script to process the results from Study 1.
# It prints the results in a variety of formats.

require 'pry'
require 'csv'

ab_index = {
  ["abortion", ["layout", "formatted"]] => ["B", "A"],
  ["abortion", ["layout", "stock"]] => ["A", "B"],
  ["abortion", ["plain", "stock"]] => ["A", "B"],
  ["creation", ["layout", "formatted"]] => ["A", "B"],
  ["creation", ["layout", "stock"]] => ["B", "A"],
  ["creation", ["plain", "stock"]] => ["B", "A"],
  ["god", ["layout", "formatted"]] => ["B", "A"],
  ["god", ["layout", "stock"]] => ["A", "B"],
  ["god", ["plain", "stock"]] => ["A", "B"],
  ["guns", ["layout", "formatted"]] => ["A", "B"],
  ["guns", ["layout", "stock"]] => ["B", "A"],
  ["guns", ["plain", "stock"]] => ["B", "A"],
  ["gay_rights", ["layout", "formatted"]] => ["B", "A"],
  ["gay_rights", ["layout", "stock"]] => ["A", "B"],
  ["gay_rights", ["plain", "stock"]] => ["A", "B"],
  ["healthcare", ["layout", "formatted"]] => ["A", "B"],
  ["healthcare", ["layout", "stock"]] => ["B", "A"],
  ["healthcare", ["plain", "stock"]] => ["B", "A"]
}

responses = []
Dir.glob('./summary_comparison/sum*') do |path|
  rows = []
  CSV.foreach(path) do |row|
    rows << row
  end
  headers, rows = rows[0], rows[1..-1]
  responses += rows.map { |r| Hash[*headers.zip(r).flatten] }
end

puts "Unique Workers: #{responses.map { |x| x["WorkerId"] }.uniq.size}"

answers = []
comments = []
responses.each do |r|
  r.select { |k, _| k.include? "Answer" }.sort_by { |_, v| v.size }.each do |q, a|
    topic, comparing, type =  q.gsub("Answer.", "").split("-")
    a = a.gsub(",", ";").gsub(/\s+/, " ")
    if a.match(/^\w+\-\w+$|^same$/)
      answers << [topic, *comparing.split("_v_"), type, a]
      last = answers.last
    else
      last_response = r.select { |k, v| k.include?(comparing + "-overall") }.first.last
      orientation = ab_index[[topic, comparing.split("_v_")]]
      translation = Hash[*orientation.zip(comparing.split("_v_")).flatten]
      comments << { topic: topic, translation: translation, comment: a, last_response: last_response }
    end
  end
end

[["plain", "stock"], ["layout", "stock"], ["layout", "formatted"]].each do |pair|
  ["overall", "content", "punctuation", "readability", "organization"].each do |factor|
    counts = answers.map { |a| a[1..-1] }
      .select { |r| r[0..1] == pair && r[2] == factor}
      .group_by { |x| x.last }
      .map { |k, v| [k, v.size] }
      .sort_by { |k, v| v }
    next if counts.empty?
    sum = counts.map(&:last).reduce(:+)
    counts = Hash[*counts.flatten]
    [pair.first+"-much_better", pair.first+"-better", "same", pair.last+"-better", pair.last+"-much_better"].each do |answer|
      puts [pair.join("_vs_"), factor, answer, counts[answer].to_i].join(",")
    end
  end
end

comments.group_by { |c| c[:translation].values.sort }.each do |k, comments|
  puts k.join(" vs. ")
  comments.each do |comment|
    puts [
      comment[:topic].upcase,
      comment[:translation].map { |k, v| "#{v} was #{k}" }.join(", "),
      "rating given: " + comment[:last_response]
    ].join(", ")
    text = comment[:comment]
    comment[:translation].each do |from, to|
      text.gsub!(/(^|\W)#{from}\W/, " [#{to.upcase}] ")
    end
    puts text.strip
    puts "-"*75
  end
  puts ["*"*100, "*"*100, "\n"].join("\n")
end


\end{minted}
\pagebreak

\section*{evaluation/r\_analysis/hists.r}
\begin{minted}[linenos=true,breaklines=true]{R}
require(ggplot2)
require(gridExtra)

library(scales)

myHist <- function(dataTable, graphTitle, xLabelsFrom, xLabelsTo, maxY){
  dataTable$per <- sapply(dataTable$count, function(x) sprintf("%1.0f%%", round(x/54, 2) * 100))
  hist <- ggplot(dataTable, aes(y=dataTable$count, x=factor(dataTable$answer, levels=unique(as.character(dataTable$answer))))) +
         geom_bar(stat="identity", fill="grey") +
         geom_text(aes(label=dataTable$per, y=dataTable$count, x=factor(dataTable$answer, levels=unique(as.character(dataTable$answer)))), vjust=-0.3, size=2.5) +
         ylab("") + xlab("") +
         theme_bw() + theme(plot.title=element_text(size=8, vjust=-3),
                            plot.margin=unit(c(0.1,0.1,-0.4,-0.4), "cm"),
                            axis.text=element_text(size=8),
                            panel.grid.major = element_blank(),
                            panel.grid.minor = element_blank()) +
         theme(axis.text.x = element_text(size=8, angle=90, hjust=1.05, vjust=0.4, lineheight=0.72)) +
         ylim(0,maxY) +
         scale_x_discrete(breaks=xLabelsFrom, labels=xLabelsTo) +
         annotate("text",  x=Inf, y = Inf, label = graphTitle, size=3, hjust=1+(1.0/nchar(graphTitle)), vjust=1.5)
  return(hist)
}

#------------------------------------------------------------------------------
# Study 1
#------------------------------------------------------------------------------

table = read.csv(file="study1.csv", header=TRUE, sep=",")
attach(table)

#------------------------------------------------------------------------------

dataLabels = c("plain-much_better", "plain-better", "same", "stock-better", "stock-much_better")
printLabels = c("plain \n much better", "plain better", "same ", "stock better", "stock \n much better")

plot1 <- myHist(table[table$comparison=='plain_vs_stock' & table$factor=='overall',],
                "overall", dataLabels, printLabels, 35)
plot2 <- myHist(table[table$comparison=='plain_vs_stock' & table$factor=='content',],
                "content", dataLabels, printLabels, 35)
plot3 <- myHist(table[table$comparison=='plain_vs_stock' & table$factor=='punctuation',],
                "punctuation", dataLabels, printLabels, 35)
plot4 <- myHist(table[table$comparison=='plain_vs_stock' & table$factor=='readability',],
                "readability", dataLabels, printLabels, 35)
plot5 <- myHist(table[table$comparison=='plain_vs_stock' & table$factor=='organization',],
                "organization", dataLabels, printLabels, 35)

pdf("plain_vs_stock_hists.pdf", width=8, height=2)
grid.arrange(plot1, plot2, plot3, plot4, plot5, ncol=5, nrow=1)

#------------------------------------------------------------------------------

dataLabels = c("layout-much_better", "layout-better", "same", "stock-better", "stock-much_better")
printLabels = c("layout \n much better", "layout better", "same ", "stock better", "stock \n much better")

plot1 <- myHist(table[table$comparison=='layout_vs_stock' & table$factor=='overall',],
                "overall", dataLabels, printLabels, 40)
plot2 <- myHist(table[table$comparison=='layout_vs_stock' & table$factor=='content',],
                "content", dataLabels, printLabels, 40)
plot3 <- myHist(table[table$comparison=='layout_vs_stock' & table$factor=='punctuation',],
                "punctuation", dataLabels, printLabels, 40)
plot4 <- myHist(table[table$comparison=='layout_vs_stock' & table$factor=='readability',],
                "readability", dataLabels, printLabels, 40)
plot5 <- myHist(table[table$comparison=='layout_vs_stock' & table$factor=='organization',],
                "organization", dataLabels, printLabels, 40)

pdf("layout_vs_stock_hists.pdf", width=8, height=2)
grid.arrange(plot1, plot2, plot3, plot4, plot5, ncol=5, nrow=1)

#------------------------------------------------------------------------------

dataLabels = c("formatted-much_better", "formatted-better", "same", "layout-better", "layout-much_better")
printLabels = c("formatted \n much better", "formatted better", "same ", "layout better", "layout \n much better")

plot1 <- myHist(table[table$comparison=='layout_vs_formatted' & table$factor=='overall',],
                "overall", dataLabels, printLabels, 35)

pdf("layout_vs_formatted_hists.pdf", width=1.6, height=2)
grid.arrange(plot1, ncol=1, nrow=1)

#------------------------------------------------------------------------------
# Study 2
#------------------------------------------------------------------------------

table = read.csv(file="study2.csv", header=TRUE, sep=",")
attach(table)

table

dataLabels = c("bigram-much_better", "bigram-better", "same", "random-better", "random-much_better")
printLabels = c("bigram \n much better", "bigram better", "same ", "random better", "random \n much better")

table[table$comparison=='bigram_vs_random' & table$factor=='overall',]

plot1 <- myHist(table[table$comparison=='bigram_vs_random' & table$factor=='overall',],
                "overall", dataLabels, printLabels, 35)

pdf("bigram_vs_random_hists.pdf", width=1.6, height=2)
grid.arrange(plot1, ncol=1, nrow=1)

dev.off()


\end{minted}
\pagebreak

\section*{evaluation/r\_analysis/scatter.r}
\begin{minted}[linenos=true,breaklines=true]{R}
require(ggplot2)
require(gridExtra)
require(psych)

library(scales)

table = read.csv(file="study2_extracts.csv", header=TRUE, sep=",")
attach(table)

table$bigram = scale(table$bigram, center=TRUE, scale=TRUE)
table$turkers = scale(table$turkers, center=TRUE, scale=TRUE)

plot <- ggplot(table, aes(x=bigram, y=turkers)) +
               geom_point(color="gray") +
               geom_smooth(method=lm, se=FALSE, color="black") +
               theme_bw() + theme(plot.title=element_text(size=8),
                                  axis.text=element_text(size=8),
                                  panel.grid.major = element_blank(),
                                  panel.grid.minor = element_blank()) +
               labs(x="Extract Bigram Score (z Score)", y="Mean Extract Human Score (z Score)")

pdf("scatter.pdf", width=4, height=4)
grid.arrange(plot, ncol=1, nrow=1)
dev.off()


\end{minted}
\pagebreak

\section*{evaluation/r\_analysis/sig\_tests.r}
\begin{minted}[linenos=true,breaklines=true]{R}
library(scales)

study1 = read.csv(file="study1.csv", header=TRUE, sep=",")
study2 = read.csv(file="study2.csv", header=TRUE, sep=",")
study2_extracts = read.csv(file="study2_extracts.csv", header=TRUE, sep=",")

successesFailures <- function(table, comparison, factor) {
  print(sprintf("total: %i",sum(table[table$comparison==comparison & table$factor==factor,]$count)))
  counts = table[table$comparison==comparison & table$factor==factor & table$answer!='same',]$count
  successes = sum(counts[1:2])
  failures = sum(counts[3:4])

  return(c(successes, failures, successes + failures))
}

study2_extracts$bigram = scale(study2_extracts$bigram, center=TRUE, scale=TRUE)
study2_extracts$turkers = scale(study2_extracts$turkers, center=TRUE, scale=TRUE)


factors = c('overall', 'content', 'punctuation', 'readability', 'organization')
for(i in 1:5) {
  succFail = successesFailures(study1, 'plain_vs_stock', factors[i])
  print(factors[i])
  print(binom.test(succFail[1], succFail[3])$p.value)
}

for(i in 1:5) {
  succFail = successesFailures(study1, 'layout_vs_stock', factors[i])
  print(factors[i])
  print(binom.test(succFail[1], succFail[3])$p.value)
}

succFail = successesFailures(study1, 'layout_vs_formatted', 'overall')
print(binom.test(succFail[1], succFail[3])$p.value)

succFail = successesFailures(study2, 'bigram_vs_random', 'overall')
print(binom.test(succFail[1], succFail[3])$p.value)

cor.test(study2_extracts$bigram, study2_extracts$turkers, short=FALSE, exact=TRUE, method="pearson", alternative="greater")


\end{minted}
\pagebreak

\section*{evaluation/survey/extracts.rb}
\begin{minted}[linenos=true,breaklines=true]{Ruby}
require 'pry'
require 'erb'

summaries = {}
Dir.glob('summaries/*') do |file|
  name = file.split(/\W/)[1]
  contents = File.open(file).read
  summaries[name] = contents
end

topics = %w(abortion creation guns god gay_rights healthcare)




groups = topics.map do |t|
  sets = File.open("extracts_#{t}.txt").read.split("------")
  sets.map { |s| s.split("\n").select { |l| l.length > 0 && l[0].match(/\w/) }.shuffle }
end

index = 0
topics.zip(groups).each do |topic, sets|
  erb = ERB.new(File.open("survey_extract.html.erb").read, 0, '>')
  File.open("#{topic}.html", "w") { |file| file.write(erb.result(binding)) }
  index += 1
end


\end{minted}
\pagebreak

\section*{evaluation/survey/generate.rb}
\begin{minted}[linenos=true,breaklines=true]{Ruby}
# generate.rb
#
# This is a script used in generating the evaluation questionnaires

require 'pry'
require 'erb'

summaries = {}
Dir.glob('summaries/*') do |file|
  name = file.split(/\W/)[1]
  contents = File.open(file).read
  summaries[name] = contents
end

sets = [%w(abortion creation god), %w(creation abortion guns), %w(god guns gay_rights),
  %w(guns god healthcare), %w(gay_rights healthcare abortion), %w(healthcare gay_rights creation)]

sets.each_with_index do |set, index|
  debate1, debate2, debate3 = set
  erb = ERB.new(File.open("survey.html.erb").read, 0, '>')
  File.open("#{index+1}.html", "w") { |file| file.write(erb.result(binding)) }
end


\end{minted}
\pagebreak

\section*{evaluation/survey/server.rb}
\begin{minted}[linenos=true,breaklines=true]{Ruby}
# server.rb
#
# This is simple test server for generated surveys. It hosts surveys in a
# similar way to mturk, allowing you to test the responses

require 'sinatra'

set :bind, '0.0.0.0'

get '/' do
  html = File.open(Dir.glob('*.html').sample).read
  "<form action=\"/\" method=\"post\">" + html + "<input type=\"submit\" value=\"Submit\">"
end

post '/' do
  "<table>" +
  params.map { |k, v| "<td>#{k}</td><td>#{v}</td>" }.join("</tr><tr>") +
  "</table>"
end


\end{minted}
\pagebreak

\section*{evaluation/survey/survey.html.erb}
\begin{minted}[linenos=true,breaklines=true]{XML+Ruby}
<!-- <%= index + 1 %> -->
<meta charset="utf-8">
<meta name="viewport" content="width=device-width, initial-scale=1.0">
<script src="https://ajax.googleapis.com/ajax/libs/jquery/2.2.2/jquery.min.js"></script>
<link rel="stylesheet" href="https://maxcdn.bootstrapcdn.com/bootstrap/3.3.6/css/bootstrap.min.css">
<style type="text/css">
  h1 {
    margin: 0.5em 0;
  }
  p {
    margin: 2px 0px 0px 0px;
    font-weight: 300;
  }
  .well {
    margin-top: 10px;
  }
  .plain blockquote, .layout blockquote {
    background: none;
    border-left: none;
    padding: 0px;
    margin: 0px 0px 1px 20px;
  }
  blockquote p {
    margin: 0;
    font-size: 14px;
    font-style: normal;
    font-family: "Source Sans Pro", sans-serif;
  }
  .plain li p {
    margin: 3px 0;
  }
  .plain ul {
    margin-bottom: 0;
  }
  .plain p, .layout p, .stock p {
    border-bottom: 1px solid #eee;
  }
  code {
    font-family: "Source Sans Pro", sans-serif;
    font-size: 1em;
    line-height: 1.45;
    background-color: transparent;
    padding: 0px;
    color: SeaGreen;
    font-weight: bold;
  }
  .plain strong, .layout strong {
    font-weight: 300;
  }
  .plain ul
  {
    list-style-type: none;
  }
  .formatted blockquote {
    margin: 0px 0px 1px 20px;
    padding: 2px 5px;
    background: #FAFAFA;
  }
  #completed {
    font-size: 1.5em;
    color: red;
    font-weight: bold;
  }
</style>
<script type="text/javascript">
  function remaining() {
    var names = [];
    var completed = 0;
    $.each($("#myForm input"), function( index, value ) {
      names.push(value.name);
      if (value.checked) {
        completed += 1
      }
    });
    $.each($("#myform textarea"), function( index, value ) {
      names.push(value.name);
      if (value.value.length > 1) {
        completed += 1
      }
    });

    var total = jQuery.unique( names ).length;
    return total - completed;
  }

  function setCompleteStatus() {
    var rem = remaining();
    $("#completed").text("Questions Remaining: " + rem.toString());
    if (rem === 0) {
      $("#completed").hide();
    } else {
      $("#completed").show();
    }
  }

  $( document ).ready(function() {
    $( "#myForm input, #myForm textarea" ).change(function() {
      setCompleteStatus();
    });
    $( "#myForm textarea" ).on('input', function() {
      setCompleteStatus();
    });
    $( "#myForm textarea" ).focusout(function() {
      setCompleteStatus();
    });
  });
</script>

<div id="myForm" class="container">
  <div class="row">
    <h1>Debate Summarization Evaluation</h1>
    <p>Please do not complete this task, <strong>Summary Comparison <%= index + 1 %></strong>, more than once.</p>
    <br/>
    <div>
      <p>The online world is full of debate and discussion - could these online debates be summarized automatically? We are investigating this question for an undergraduate research project and are asking people to compare different summaries. Making an effort to remain indifferent to the topics, please complete the form below to let us know what you think.</p>
      <p>Please contact <a href="mailto:c.egan.12@aberdeen.ac.uk">c.egan.12@aberdeen.ac.uk</a> if you have any queries before starting the survey.</p>
    </div>
    <h3>Consent Form</h3>
    <div class="well">
      <p>Please read the statements below</p>
      <ul>
        <li>I confirm that the research project <strong>Debate Summarization Tool Evaluation</strong> has been explained to me. I have had the opportunity to ask questions about the project and have had these answered satisfactorily.</li>
        <li>I consent to the material I contribute being used to generate insights for the research project <strong>Debate Summarization Tool Evaluation</strong>.</li>
        <li>I understand that my participation in this research is voluntary and that I may withdraw from the study at any time.</li>
        <li>I consent to allow the <strong>fully anonymized</strong> data to be used for future publications and other scholarly means of disseminating the findings from the research project.</li>
        <li>I understand that the information/data acquired will be securely stored by researchers, but that appropriately anonymized data may in future be made available to others for research purposes only.</li>
      </ul>
      <p><strong>By continuing this questionnaire you agree to the above statements.</strong></p>
    </div>
  </div>

  <hr/>

  <div class="row">
    <h3>Instructions</h3>
    <div>
      <p>We will ask you to compare two summaries side-by-side, that differ in content or presentation, of the same online debate. </p>
      <p>The task is just about the comparison of summaries at face value since reading the long source text would take too long. </p>
      <p>Please remember that this is a student research project and that summaries are not intended to give preference to a particular viewpoint, nor should you allow your personal stance on a topic influence your decision.</p>
      <p>In the following sections you will be asked to compare summaries on the following factors:</p>
      <ul>
        <li><strong>Content Interest / Informativeness</strong> the summary presents varied and interesting content.</li>
        <li><strong>Readability</strong> the summary contents make sense; work without context; aren't repetitive; and are easy to read.</li>
        <li><strong>Punctuation and Presentation</strong> the summary contents are correctly formatted as sentences, punctuation, capital letters and have sentence case.</li>
        <li><strong>Organization</strong> related points occur near one and other.</li>
        <li><strong>Overall</strong> aggregating your scores from those listed above - <strong>and any other factors you deem relevant</strong> - select your preferred summary.</li>
      </ul>
      <p>Please also note that lines in summaries are to be read independently (they are not paragraphs).</p>
      <p>A summary is <strong>not</strong> intended to present a structured argument on the topic but rather <strong>give you a feel</strong> for what people talk about.</p>
    </div>
  </div>

  <hr/>

  <h2>Section 1</h2>
  <p>Please read the following summaries:</p>
  <% s1, s2 = [['plain', summaries[debate1+"_plain"]], ['stock', summaries[debate1+"_stock_short"]]] %>
  <% s1, s2 = [s1, s2].reverse if index.odd? %>
  <div class="row">
    <div class="col-md-6 <%= s1.first%>">
      <h4><%= debate1 %> A</h4>
      <%= s1.last %>
    </div>
    <div class="col-md-6 <%= s2.first%>">
      <h4><%= debate1 %> B</h4>
      <%= s2.last %>
    </div>
  </div>
  <div class="well">
    <p>1.1. Content Interest / Informativeness</p>
    <label class="radio-inline"><input type="radio" name="<%= debate1 %>-plain_v_stock-content" value="<%= s1.first %>-much_better">A is much better</label>
    <label class="radio-inline"><input type="radio" name="<%= debate1 %>-plain_v_stock-content" value="<%= s1.first %>-better">A is better</label>
    <label class="radio-inline"><input type="radio" name="<%= debate1 %>-plain_v_stock-content" value="same">A and B are about the same</label>
    <label class="radio-inline"><input type="radio" name="<%= debate1 %>-plain_v_stock-content" value="<%= s2.first %>-better">B is better</label>
    <label class="radio-inline"><input type="radio" name="<%= debate1 %>-plain_v_stock-content" value="<%= s2.first %>-much_better">B is much better</label>
    <p>1.2. Readability</p>
    <label class="radio-inline"><input type="radio" name="<%= debate1 %>-plain_v_stock-readability" value="<%= s1.first %>-much_better">A is much better</label>
    <label class="radio-inline"><input type="radio" name="<%= debate1 %>-plain_v_stock-readability" value="<%= s1.first %>-better">A is better</label>
    <label class="radio-inline"><input type="radio" name="<%= debate1 %>-plain_v_stock-readability" value="same">A and B are about the same</label>
    <label class="radio-inline"><input type="radio" name="<%= debate1 %>-plain_v_stock-readability" value="<%= s2.first %>-better">B is better</label>
    <label class="radio-inline"><input type="radio" name="<%= debate1 %>-plain_v_stock-readability" value="<%= s2.first %>-much_better">B is much better</label>
    <p>1.3. Punctuation and Presentation</p>
    <label class="radio-inline"><input type="radio" name="<%= debate1 %>-plain_v_stock-punctuation" value="<%= s1.first %>-much_better">A is much better</label>
    <label class="radio-inline"><input type="radio" name="<%= debate1 %>-plain_v_stock-punctuation" value="<%= s1.first %>-better">A is better</label>
    <label class="radio-inline"><input type="radio" name="<%= debate1 %>-plain_v_stock-punctuation" value="same">A and B are about the same</label>
    <label class="radio-inline"><input type="radio" name="<%= debate1 %>-plain_v_stock-punctuation" value="<%= s2.first %>-better">B is better</label>
    <label class="radio-inline"><input type="radio" name="<%= debate1 %>-plain_v_stock-punctuation" value="<%= s2.first %>-much_better">B is much better</label>
    <p>1.4. Organization</p>
    <label class="radio-inline"><input type="radio" name="<%= debate1 %>-plain_v_stock-organization" value="<%= s1.first %>-much_better">A is much better</label>
    <label class="radio-inline"><input type="radio" name="<%= debate1 %>-plain_v_stock-organization" value="<%= s1.first %>-better">A is better</label>
    <label class="radio-inline"><input type="radio" name="<%= debate1 %>-plain_v_stock-organization" value="same">A and B are about the same</label>
    <label class="radio-inline"><input type="radio" name="<%= debate1 %>-plain_v_stock-organization" value="<%= s2.first %>-better">B is better</label>
    <label class="radio-inline"><input type="radio" name="<%= debate1 %>-plain_v_stock-organization" value="<%= s2.first %>-much_better">B is much better</label>
    <p>1.5. Overall</p>
    <label class="radio-inline"><input type="radio" name="<%= debate1 %>-plain_v_stock-overall" value="<%= s1.first %>-much_better">A is much better</label>
    <label class="radio-inline"><input type="radio" name="<%= debate1 %>-plain_v_stock-overall" value="<%= s1.first %>-better">A is better</label>
    <label class="radio-inline"><input type="radio" name="<%= debate1 %>-plain_v_stock-overall" value="same">A and B are about the same</label>
    <label class="radio-inline"><input type="radio" name="<%= debate1 %>-plain_v_stock-overall" value="<%= s2.first %>-better">B is better</label>
    <label class="radio-inline"><input type="radio" name="<%= debate1 %>-plain_v_stock-overall" value="<%= s2.first %>-much_better">B is much better</label>
    <p>1.6. Please justify your answer to Question 1.5:</p>
    <textarea class="form-control" rows="5" name="<%= debate1 %>-plain_v_stock-comment"></textarea>
  </div>

  <hr/>

  <h2>Section 2</h2>
  <p>Please read the following summaries:</p>
  <% s1, s2 = [['layout', summaries[debate2+"_layout"]], ['stock', summaries[debate2+"_stock_long"]]] %>
  <% s1, s2 = [s1, s2].reverse if index.even? %>
  <div class="row">
    <div class="col-md-6 <%= s1.first%>">
      <h4><%= debate2 %> A</h4>
      <%= s1.last %>
    </div>
    <div class="col-md-6 <%= s2.first%>">
      <h4><%= debate2 %> B</h4>
      <%= s2.last %>
    </div>
  </div>
  <div class="well">
    <p>2.1. Content Interest / Informativeness</p>
    <label class="radio-inline"><input type="radio" name="<%= debate2 %>-layout_v_stock-content" value="<%= s1.first %>-much_better">A is much better</label>
    <label class="radio-inline"><input type="radio" name="<%= debate2 %>-layout_v_stock-content" value="<%= s1.first %>-better">A is better</label>
    <label class="radio-inline"><input type="radio" name="<%= debate2 %>-layout_v_stock-content" value="same">A and B are about the same</label>
    <label class="radio-inline"><input type="radio" name="<%= debate2 %>-layout_v_stock-content" value="<%= s2.first %>-better">B is better</label>
    <label class="radio-inline"><input type="radio" name="<%= debate2 %>-layout_v_stock-content" value="<%= s2.first %>-much_better">B is much better</label>
    <p>2.2. Readability</p>
    <label class="radio-inline"><input type="radio" name="<%= debate2 %>-layout_v_stock-readability" value="<%= s1.first %>-much_better">A is much better</label>
    <label class="radio-inline"><input type="radio" name="<%= debate2 %>-layout_v_stock-readability" value="<%= s1.first %>-better">A is better</label>
    <label class="radio-inline"><input type="radio" name="<%= debate2 %>-layout_v_stock-readability" value="same">A and B are about the same</label>
    <label class="radio-inline"><input type="radio" name="<%= debate2 %>-layout_v_stock-readability" value="<%= s2.first %>-better">B is better</label>
    <label class="radio-inline"><input type="radio" name="<%= debate2 %>-layout_v_stock-readability" value="<%= s2.first %>-much_better">B is much better</label>
    <p>2.3. Punctuation and Presentation</p>
    <label class="radio-inline"><input type="radio" name="<%= debate2 %>-layout_v_stock-punctuation" value="<%= s1.first %>-much_better">A is much better</label>
    <label class="radio-inline"><input type="radio" name="<%= debate2 %>-layout_v_stock-punctuation" value="<%= s1.first %>-better">A is better</label>
    <label class="radio-inline"><input type="radio" name="<%= debate2 %>-layout_v_stock-punctuation" value="same">A and B are about the same</label>
    <label class="radio-inline"><input type="radio" name="<%= debate2 %>-layout_v_stock-punctuation" value="<%= s2.first %>-better">B is better</label>
    <label class="radio-inline"><input type="radio" name="<%= debate2 %>-layout_v_stock-punctuation" value="<%= s2.first %>-much_better">B is much better</label>
    <p>2.4. Organization</p>
    <label class="radio-inline"><input type="radio" name="<%= debate2 %>-layout_v_stock-organization" value="<%= s1.first %>-much_better">A is much better</label>
    <label class="radio-inline"><input type="radio" name="<%= debate2 %>-layout_v_stock-organization" value="<%= s1.first %>-better">A is better</label>
    <label class="radio-inline"><input type="radio" name="<%= debate2 %>-layout_v_stock-organization" value="same">A and B are about the same</label>
    <label class="radio-inline"><input type="radio" name="<%= debate2 %>-layout_v_stock-organization" value="<%= s2.first %>-better">B is better</label>
    <label class="radio-inline"><input type="radio" name="<%= debate2 %>-layout_v_stock-organization" value="<%= s2.first %>-much_better">B is much better</label>
    <p>2.5. Overall</p>
    <label class="radio-inline"><input type="radio" name="<%= debate2 %>-layout_v_stock-overall" value="<%= s1.first %>-much_better">A is much better</label>
    <label class="radio-inline"><input type="radio" name="<%= debate2 %>-layout_v_stock-overall" value="<%= s1.first %>-better">A is better</label>
    <label class="radio-inline"><input type="radio" name="<%= debate2 %>-layout_v_stock-overall" value="same">A and B are about the same</label>
    <label class="radio-inline"><input type="radio" name="<%= debate2 %>-layout_v_stock-overall" value="<%= s2.first %>-better">B is better</label>
    <label class="radio-inline"><input type="radio" name="<%= debate2 %>-layout_v_stock-overall" value="<%= s2.first %>-much_better">B is much better</label>
    <p>2.6. Please justify your answer to 2.5:</p>
    <textarea class="form-control" rows="5" name="<%= debate2 %>-layout_v_stock-comment"></textarea>
  </div>

  <hr/>

  <h2>Section 3</h2>
  <p>Please review the following summaries considering the presentation, note that you have already read this summary in the previous question.</p>
  <p>In the formatted summary, keywords in explanations are in bold; topic words are in green bold and summary content has a grey background.</p>
  <% s1, s2 = [['layout', summaries[debate2+"_layout"]], ['formatted', summaries[debate2+"_formatted"]]] %>
  <% s1, s2 = [s1, s2].reverse if index.odd? %>
  <div class="row">
    <div class="col-md-6 <%= s1.first%>">
      <h4><%= debate2 %> A</h4>
      <%= s1.last %>
    </div>
    <div class="col-md-6 <%= s2.first%>">
      <h4><%= debate2 %> B</h4>
      <%= s2.last %>
    </div>
  </div>
  <div class="well">
    <p>3.1. Which presentation do you prefer?<p>
    <label class="radio-inline"><input type="radio" name="<%= debate2 %>-layout_v_formatted-overall" value="<%= s1.first %>-much_better">A is much better</label>
    <label class="radio-inline"><input type="radio" name="<%= debate2 %>-layout_v_formatted-overall" value="<%= s1.first %>-better">A is better</label>
    <label class="radio-inline"><input type="radio" name="<%= debate2 %>-layout_v_formatted-overall" value="same">A and B are about the same</label>
    <label class="radio-inline"><input type="radio" name="<%= debate2 %>-layout_v_formatted-overall" value="<%= s2.first %>-better">B is better</label>
    <label class="radio-inline"><input type="radio" name="<%= debate2 %>-layout_v_formatted-overall" value="<%= s2.first %>-much_better">B is much better</label>
    <p>3.2. Please justify your answer to 3.1:</p>
    <textarea class="form-control" rows="5" name="<%= debate2 %>-layout_v_formatted-comment"></textarea>
  </div>
  <p id="completed">Please answer all the questions.</p>
</div>


\end{minted}
\pagebreak

\section*{evaluation/survey/survey\_extract.html.erb}
\begin{minted}[linenos=true,breaklines=true]{XML+Ruby}
<!-- <%= index + 1 %> -->
<meta charset="utf-8">
<meta name="viewport" content="width=device-width, initial-scale=1.0">
<script src="https://ajax.googleapis.com/ajax/libs/jquery/2.2.2/jquery.min.js"></script>
<link rel="stylesheet" href="https://maxcdn.bootstrapcdn.com/bootstrap/3.3.6/css/bootstrap.min.css">
<style type="text/css">
  h1 {
    margin: 0.5em 0;
  }
  p {
    margin: 2px 0px 0px 0px;
    font-weight: 300;
  }
  .well {
    margin-top: 10px;
  }
  .plain blockquote, .layout blockquote {
    background: none;
    border-left: none;
    padding: 0px;
    margin: 0px 0px 1px 20px;
  }
  blockquote p {
    margin: 0;
    font-size: 14px;
    font-style: normal;
    font-family: "Source Sans Pro", sans-serif;
  }
  .plain li p {
    margin: 3px 0;
  }
  .plain ul {
    margin-bottom: 0;
  }
  .plain p, .layout p, .stock p {
    border-bottom: 1px solid #eee;
  }
  code {
    font-family: "Source Sans Pro", sans-serif;
    font-size: 1em;
    line-height: 1.45;
    background-color: transparent;
    padding: 0px;
    color: SeaGreen;
    font-weight: bold;
  }
  .plain strong, .layout strong {
    font-weight: 300;
  }
  .plain ul
  {
    list-style-type: none;
  }
  .formatted blockquote {
    margin: 0px 0px 1px 20px;
    padding: 2px 5px;
    background: #FAFAFA;
  }
  #completed {
    font-size: 1.5em;
    color: red;
    font-weight: bold;
  }
</style>
<script type="text/javascript">
  function remaining() {
    var names = [];
    var completed = 0;
    $.each($("#myForm input"), function( index, value ) {
      names.push(value.name);
      if (value.checked) {
        completed += 1
      }
    });
    $.each($("#myform textarea"), function( index, value ) {
      names.push(value.name);
      if (value.value.length > 1) {
        completed += 1
      }
    });

    var total = jQuery.unique( names ).length;
    return total - completed;
  }

  function setCompleteStatus() {
    var rem = remaining();
    $("#completed").text("Questions Remaining: " + rem.toString());
    if (rem === 0) {
      $("#completed").hide();
    } else {
      $("#completed").show();
    }
  }

  $( document ).ready(function() {
    $( "#myForm input, #myForm textarea" ).change(function() {
      setCompleteStatus();
    });
    $( "#myForm textarea" ).on('input', function() {
      setCompleteStatus();
    });
    $( "#myForm textarea" ).focusout(function() {
      setCompleteStatus();
    });
  });
</script>

<div id="myForm" class="container">
  <div class="row">
    <h1>Debate Summarization Evaluation</h1>
    <p>Please do not complete this task, <strong>Extracts <%= index + 1 %></strong>, more than once.</p>
    <br/>
    <div>
      <p>The online world is full of debate and discussion - could these online debates be summarized automatically? We are investigating this question for an undergraduate research project and are comparing some extracts from debates for use in summaries. Complete the form below to let us know what you think.</p>
      <p>Please contact <a href="mailto:c.egan.12@aberdeen.ac.uk">c.egan.12@aberdeen.ac.uk</a> if you have any queries before starting the survey.</p>
    </div>
    <h3>Consent Form</h3>
    <div class="well">
      <p>Please read the statements below</p>
      <ul>
        <li>I confirm that the research project <strong>Debate Summarization Tool Evaluation</strong> has been explained to me. I have had the opportunity to ask questions about the project and have had these answered satisfactorily.</li>
        <li>I consent to the material I contribute being used to generate insights for the research project <strong>Debate Summarization Tool Evaluation</strong>.</li>
        <li>I understand that my participation in this research is voluntary and that I may withdraw from the study at any time.</li>
        <li>I consent to allow the <strong>fully anonymized</strong> data to be used for future publications and other scholarly means of disseminating the findings from the research project.</li>
        <li>I understand that the information/data acquired will be securely stored by researchers, but that appropriately anonymized data may in future be made available to others for research purposes only.</li>
      </ul>
      <p><strong>By continuing this questionnaire you agree to the above statements.</strong></p>
    </div>
  </div>

  <hr/>

  <div class="row">
    <h3>Instructions</h3>
    <div>
      <p>We have built a tool to extract points made in a discussion on <strong><%= topic.gsub("_", " ") %></strong>.</p>
      <p>Given the following sets of similar points, rate them accounting for their <strong>succinctness</strong> and the extent to which they <strong>make sense</strong>.</p>
      <br/>
      <ul>
        <li>Example of a <strong>good</strong> extract: <em>The bill was approved.</em></li>
        <li>Example of a <strong>worse</strong> extract: <em>Similarly, the bill was approved.</em></li>
        <li>Example of a <strong>bad</strong> extract: <em>The bill was approved as the.</em></li>
      </ul>
    </div>

    Higher scores like 5/5 are better than lower ones like 1/5 or 2/5.
  </div>

  <hr/>

  <div class="row">
    <h2>Section 1</h2>
    <% sets.each_with_index do |set, set_index| %>
      <h4>Set <%= set_index + 1 %></h4>
      <div class="well">
      <% set.each_with_index do |e, i| %>
        <% i += 1 %>
        <p><strong>1.<%= set_index + 1 %>.<%= "#{i}." %></strong>  <%= e %></p>
        <label class="radio-inline"><input type="radio" name="<%= topic %>-set<%= set_index + 1 %>extract<%= i %>" value="1">1/5</label>
        <label class="radio-inline"><input type="radio" name="<%= topic %>-set<%= set_index + 1 %>extract<%= i %>" value="2">2/5</label>
        <label class="radio-inline"><input type="radio" name="<%= topic %>-set<%= set_index + 1 %>extract<%= i %>" value="3">3/5</label>
        <label class="radio-inline"><input type="radio" name="<%= topic %>-set<%= set_index + 1 %>extract<%= i %>" value="4">4/5</label>
        <label class="radio-inline"><input type="radio" name="<%= topic %>-set<%= set_index + 1 %>extract<%= i %>" value="5">5/5</label>
      <% end %>
      </div>
    <% end %>
  </div>

  <h2>Section 2</h2>
  <p>Please review the following summaries and rate them accounting for their <strong>informativeness</strong> and the extent to which they <strong>make sense</strong>.</p>
  <% s1, s2 = [['layout', summaries[topic + "_layout"]], ['random_layout', summaries['random_' + topic + "_layout"]]] %>
  <div class="row">
    <div class="col-md-6 <%= s1.first.split("_").join(" ") %>">
      <h4><%= topic %> A</h4>
      <%= s1.last %>
    </div>
    <div class="col-md-6 <%= s2.first.split("_").join(" ") %>">
      <h4><%= topic %> B</h4>
      <%= s2.last %>
    </div>
  </div>
  <div class="well">
    <p>2.1. Comparing the summary on how well, on average, the points make sense and their informativeness, give a relative rating:</p>
    <label class="radio-inline"><input type="radio" name="<%= topic %>-layout_v_layout_random" value="<%= s1.first %>-much_better">A is much better</label>
    <label class="radio-inline"><input type="radio" name="<%= topic %>-layout_v_layout_random" value="<%= s1.first %>-better">A is better</label>
    <label class="radio-inline"><input type="radio" name="<%= topic %>-layout_v_layout_random" value="same">A and B are about the same</label>
    <label class="radio-inline"><input type="radio" name="<%= topic %>-layout_v_layout_random" value="<%= s2.first %>-better">B is better</label>
    <label class="radio-inline"><input type="radio" name="<%= topic %>-layout_v_layout_random" value="<%= s2.first %>-much_better">B is much better</label>
    <p>2.2. Please justify your answer to 2.1:</p>
    <textarea class="form-control" rows="5" name="<%= topic %>-layout_v_layout_random-comment"></textarea>
  </div>
  <p id="completed">Please answer all the questions.</p>
</div>


\end{minted}
\pagebreak

\section*{points\_api/app.rb}
\begin{minted}[linenos=true,breaklines=true]{Ruby}
# app.rb
#
# This file implements the core of the points extraction service. It defines
# a root route that that accepts requests for points analysis. It splits the
# text into chunks and extracts sentences before triggering the points analysis.

require 'json'
require 'net/http'

require 'sinatra'
require 'neo4j'
require 'pry'

require './lib/corenlp_client'
require './lib/neo4j_client'
require './lib/frame'
require './lib/node'
require './lib/relation'
require './lib/points_extraction'
require './lib/utils'

set :port, ENV['PORT']
set :bind, '0.0.0.0'
set :public_folder, 'static'

# configure the clients for communicating with the other services
corenlp_client = CoreNlpClient.new("http://corenlp_server:#{ENV['CNLP_PORT']}")
neo4j_client = Neo4jClient.new("http://neo4j:7474")

frames = JSON.parse(File.open('verbs.json', 'r').read)
frame_queries = Hash[*Dir.glob('frame_queries/*.cql').map do |path|
  [path.scan(/\/((\w|-)+)\./)[0][0].humanize.upcase.gsub('-COPULA', '-cop'),
    File.open(path, 'r').read]
end.flatten]

# requests for points analysis are handled here
post '/' do
  data = JSON.parse(request.body.read)

  points = []
  # text is split into smaller sections, CoreNLP has a 100000 char limit
  Utils.chunk_text(35000, Utils.clean_text(data["text"])).each do |text|
    # remove all the content from the database
    neo4j_client.clear
    # get parses for the sentences
    sentences = corenlp_client.request_parse(text)
    # select sentences with topic words
    sentences.select! { |s| Utils.sentence_contains_topic(s, data['topics']) }
    next if sentences.empty?
    begin
      puts sentences.size
      sentences.each_slice(5).each do |group|
        # save the sentences to the database
        query_string = neo4j_client.generate_create_query_for_sentences(group)
        neo4j_client.execute(query_string)
      end
      # find valid points
      matches = PointsExtraction.matches_for_verbs(neo4j_client, frames, frame_queries)
      # find the extracts for the valid points
      points += PointsExtraction.points_for_matches(neo4j_client, matches, data['topics'], data['keys'])
        .uniq
        .sort_by(&:size)
    rescue Exception => ex
      puts ex
      return points.to_json
    end
  end

  points.to_json
end


\end{minted}
\pagebreak

\section*{points\_api/frame\_queries/LEX\_VERB\_NP\_PREP\_NP.cql}
\begin{minted}[linenos=true,breaklines=true]{Cypher}
//LEX VERB NP PREP NP

MATCH (verb:Node)
WHERE ANY (uuid IN ["UUIDS"] WHERE uuid = verb.uuid)
MATCH (verb)-[rel_lex:REL]->(lex:Node)
MATCH (verb)-[rel_dobj:REL]->(dobj:Node)
MATCH (verb)-[rel_nmod:REL]->(nmod:Node)
MATCH (nmod)-[rel_prep:REL]->(prep:Node)
WHERE rel_lex.label = "expl"
AND rel_dobj.label = "dobj"
AND rel_nmod.label =~ 'nmod:.*'
AND rel_prep.label = "case"
RETURN lex, verb, dobj, prep, nmod;


\end{minted}
\pagebreak

\section*{points\_api/frame\_queries/LEX\_VERB\_PREP\_NP\_NP.cql}
\begin{minted}[linenos=true,breaklines=true]{Cypher}
//LEX VERB PREP NP NP

MATCH (verb:Node)
WHERE ANY (uuid IN ["UUIDS"] WHERE uuid = verb.uuid)
MATCH (verb)-[rel_lex:REL]->(lex:Node)
MATCH (verb)-[rel_nmod:REL]->(nmod:Node)
MATCH (nmod)-[rel_prep:REL]->(prep:Node)
MATCH (nmod)-[rel_dep:REL]->(dep:Node)
WHERE rel_lex.label = "expl"
AND rel_nmod.label =~ 'nmod:.*'
AND rel_prep.label = "case"
AND rel_dep.label = "dep"
RETURN lex, verb, prep, nmod, dep;


\end{minted}
\pagebreak

\section*{points\_api/frame\_queries/NP\_VERB.cql}
\begin{minted}[linenos=true,breaklines=true]{Cypher}
//NP VERB

MATCH (verb:Node)
WHERE ANY (uuid IN ["UUIDS"] WHERE uuid = verb.uuid)
MATCH (verb)-[rel_nsubj:REL]->(nsubj:Node)
WHERE rel_nsubj.label = "nsubj"
RETURN nsubj, verb;


\end{minted}
\pagebreak

\section*{points\_api/frame\_queries/NP\_VERB\_ADJ-copula.cql}
\begin{minted}[linenos=true,breaklines=true]{Cypher}
//NP VERB ADJ (copula)

MATCH (verb:Node)
WHERE ANY (uuid IN ["UUIDS"] WHERE uuid = verb.uuid)
MATCH (verb)-[rel_nsubj:REL]->(nsubj:Node)
MATCH (verb)-[rel_adj:REL]->(adj:Node)
WHERE rel_nsubj.label = "nsubj"
AND rel_adj.label = "xcomp"
RETURN nsubj, verb, adj;


\end{minted}
\pagebreak

\section*{points\_api/frame\_queries/NP\_VERB\_ADV-copula.cql}
\begin{minted}[linenos=true,breaklines=true]{Cypher}
//NP VERB ADV (copula)

MATCH (verb:Node)
WHERE ANY (uuid IN ["UUIDS"] WHERE uuid = verb.uuid)
MATCH (verb)-[rel_nsubj:REL]->(nsubj:Node)
MATCH (verb)-[rel_advmod:REL]->(adv:Node)
WHERE rel_nsubj.label = "nsubj"
AND rel_advmod.label = "advmod"
RETURN nsubj, verb, adv;


\end{minted}
\pagebreak

\section*{points\_api/frame\_queries/NP\_VERB\_ADV.cql}
\begin{minted}[linenos=true,breaklines=true]{Cypher}
//NP VERB ADV

MATCH (verb:Node)
WHERE ANY (uuid IN ["UUIDS"] WHERE uuid = verb.uuid)
MATCH (verb)-[rel_nsubj:REL]->(nsubj:Node)
MATCH (verb)-[rel_adv:REL]->(adv:Node)

WHERE rel_nsubj.label = "nsubj"
AND rel_adv.label = "advmod"

RETURN nsubj, verb, adv;


\end{minted}
\pagebreak

\section*{points\_api/frame\_queries/NP\_VERB\_ADV\_LEX.cql}
\begin{minted}[linenos=true,breaklines=true]{Cypher}
//NP VERB ADV LEX

MATCH (verb:Node)
WHERE ANY (uuid IN ["UUIDS"] WHERE uuid = verb.uuid)
MATCH (verb)-[rel_nsubj:REL]->(nsubj:Node)
MATCH (verb)-[rel_dobj:REL]->(dobj:Node)
MATCH (verb)-[rel_adv:REL]->(adv:Node)
MATCH (verb)-[rel_lex:REL]->(lex:Node)

WHERE rel_nsubj.label = "nsubj"
AND rel_dobj.label = "dobj"
AND rel_adv.label = "advmod"
AND rel_lex.label = "advmod"
AND adv.index < lex.index
RETURN nsubj, verb, dobj, lex

ORDER BY nsubj.index;


\end{minted}
\pagebreak

\section*{points\_api/frame\_queries/NP\_VERB\_ADV\_PREP\_NP.cql}
\begin{minted}[linenos=true,breaklines=true]{Cypher}
//NP VERB ADV PREP NP

MATCH (verb:Node)
WHERE ANY (uuid IN ["UUIDS"] WHERE uuid = verb.uuid)
MATCH (verb)-[rel_nsubj:REL]->(nsubj:Node)
MATCH (verb)-[rel_adv:REL]->(adv:Node)
MATCH (verb)-[rel_nmod:REL]->(nmod:Node)
MATCH (nmod)-[rel_prep:REL]->(prep:Node)

WHERE rel_nsubj.label = "nsubj"
AND rel_adv.label = "advmod"
AND rel_nmod.label =~ 'nmod.*'
AND rel_prep.label = "case"

RETURN nsubj, verb, adv;


\end{minted}
\pagebreak

\section*{points\_api/frame\_queries/NP\_VERB\_LEX.cql}
\begin{minted}[linenos=true,breaklines=true]{Cypher}
//NP VERB LEX

MATCH (verb:Node)
WHERE ANY (uuid IN ["UUIDS"] WHERE uuid = verb.uuid)
MATCH path=(nsubj:Node)-[REL*1..3]->(verb)
MATCH (np2)-[rel_lex:REL]->(lex:Node)

WHERE rel_lex.label = "advmod"
AND "acl" IN CypherRACT(r in relationships(path) | r.label)
RETURN nsubj, verb, lex
ORDER BY nsubj.index;


\end{minted}
\pagebreak

\section*{points\_api/frame\_queries/NP\_VERB\_NP-copula.cql}
\begin{minted}[linenos=true,breaklines=true]{Cypher}
//NP VERB NP (copula)

MATCH (dobj:Node)
MATCH (dobj)-[rel_cop:REL]->(verb:Node)
MATCH (dobj)-[rel_nsubj:REL]->(nsubj:Node)
WHERE rel_nsubj.label = "nsubj"
AND rel_cop.label = "cop"
AND ANY (uuid IN ["UUIDS"] WHERE uuid = verb.uuid)
RETURN nsubj, verb, dobj;


\end{minted}
\pagebreak

\section*{points\_api/frame\_queries/NP\_VERB\_NP.cql}
\begin{minted}[linenos=true,breaklines=true]{Cypher}
//NP VERB NP

MATCH (verb:Node)
WHERE ANY (uuid IN ["UUIDS"] WHERE uuid = verb.uuid)
MATCH (verb)-[rel_nsubj:REL]->(nsubj:Node)
MATCH (verb)-[rel_dobj:REL]->(dobj:Node)
WHERE rel_nsubj.label = "nsubj"
AND rel_dobj.label = "dobj"
RETURN nsubj, verb, dobj;


\end{minted}
\pagebreak

\section*{points\_api/frame\_queries/NP\_VERB\_NP\_LEX.cql}
\begin{minted}[linenos=true,breaklines=true]{Cypher}
//NP VERB NP LEX
//in conj 2nd np the first nn is selected (since it relates to the 2nd via conj:and or other)

MATCH (verb:Node)
WHERE ANY (uuid IN ["UUIDS"] WHERE uuid = verb.uuid)
MATCH (verb)-[rel_nsubj:REL]->(nsubj:Node)
MATCH (verb)-[rel_dobj:REL]->(dobj:Node)
MATCH (verb)-[rel_lex:REL]->(lex:Node)
WHERE rel_nsubj.label = "nsubj"
AND rel_dobj.label = "dobj"
AND rel_lex.label = "advmod"
RETURN nsubj, verb, dobj, lex
ORDER BY dobj.index;


\end{minted}
\pagebreak

\section*{points\_api/frame\_queries/NP\_VERB\_NP\_LEX\_NP.cql}
\begin{minted}[linenos=true,breaklines=true]{Cypher}
//NP VERB NP LEX NP

MATCH (verb:Node)
WHERE ANY (uuid IN ["UUIDS"] WHERE uuid = verb.uuid)
MATCH (verb)-[rel_nsubj:REL]->(nsubj:Node)
MATCH (verb)-[rel_dobj:REL]->(dobj:Node)
MATCH (verb)-[rel_np2:REL]->(np2:Node)
MATCH (np2)-[rel_lex:REL]->(lex:Node)

WHERE rel_nsubj.label = "nsubj"
AND rel_dobj.label = "dobj"
AND rel_np2.label = "advcl"
AND rel_lex.label = "mark"
RETURN nsubj, verb, dobj, lex, np2;


\end{minted}
\pagebreak

\section*{points\_api/frame\_queries/NP\_VERB\_NP\_NP.cql}
\begin{minted}[linenos=true,breaklines=true]{Cypher}
//NP VERB NP NP

MATCH (verb:Node)
WHERE ANY (uuid IN ["UUIDS"] WHERE uuid = verb.uuid)
MATCH (verb)-[rel_nsubj:REL]->(nsubj:Node)
MATCH (verb)-[rel_dobj:REL]->(dobj:Node)
MATCH (verb)-[rel_xcomp:REL]->(xcomp:Node)
WHERE rel_nsubj.label = "nsubj"
AND rel_dobj.label = "dobj"
AND rel_xcomp.label = "xcomp"
RETURN nsubj, verb, dobj, xcomp;


\end{minted}
\pagebreak

\section*{points\_api/frame\_queries/NP\_VERB\_NP\_PREP\_NP.cql}
\begin{minted}[linenos=true,breaklines=true]{Cypher}
//NP VERB NP PREP NP

MATCH (verb:Node)
WHERE ANY (uuid IN ["UUIDS"] WHERE uuid = verb.uuid)
MATCH (verb)-[rel_nsubj:REL]->(nsubj:Node)
MATCH (verb)-[rel_dobj:REL]->(dobj:Node)
MATCH (verb)-[rel_np3:REL]->(np3:Node)
MATCH (np3)-[rel_prep:REL]->(prep:Node)
WHERE rel_nsubj.label = "nsubj"
AND rel_dobj.label = "dobj"
AND rel_np3.label =~ 'nmod.*'
AND rel_prep.label = "case"
RETURN nsubj, verb, dobj, prep, np3;


\end{minted}
\pagebreak

\section*{points\_api/frame\_queries/NP\_VERB\_NP\_PREP\_NP\_PREP\_NP.cql}
\begin{minted}[linenos=true,breaklines=true]{Cypher}
//NP VERB NP PREP NP PREP NP

MATCH (verb:Node)
WHERE ANY (uuid IN ["UUIDS"] WHERE uuid = verb.uuid)
MATCH (verb)-[rel_nsubj:REL]->(nsubj:Node)
MATCH (verb)-[rel_dobj:REL]->(dobj:Node)
MATCH (verb)-[rel_nmod1:REL]->(nmod1:Node)
MATCH (verb)-[rel_nmod2:REL]->(nmod2:Node)
MATCH (nmod1)-[rel_case1:REL]->(prep1:Node)
MATCH (nmod2)-[rel_case2:REL]->(prep2:Node)

WHERE rel_nsubj.label = "nsubj"
AND rel_dobj.label = "dobj"
AND rel_nmod1.label =~ 'nmod.*'
AND rel_nmod2.label =~ 'nmod.*'
AND rel_case1.label = "case"
AND rel_case2.label = "case"
AND nmod1.index < nmod2.index

RETURN nsubj, verb, dobj, prep1, nmod1, prep2, nmod2;


\end{minted}
\pagebreak

\section*{points\_api/frame\_queries/NP\_VERB\_PREP\_NP.cql}
\begin{minted}[linenos=true,breaklines=true]{Cypher}
//NP VERB PREP NP

MATCH (verb:Node)
WHERE ANY (uuid IN ["UUIDS"] WHERE uuid = verb.uuid)
MATCH (verb)-[rel_nsubj:REL]->(nsubj:Node)
MATCH (verb)-[rel_dobj:REL]->(dobj:Node)
MATCH (dobj)-[rel_prep:REL]->(prep:Node)
WHERE rel_nsubj.label = "nsubj"
AND rel_dobj.label =~ 'nmod.*'
AND rel_prep.label = "case"
RETURN nsubj, verb, prep, dobj;


\end{minted}
\pagebreak

\section*{points\_api/frame\_queries/NP\_VERB\_PREP\_NP\_NP.cql}
\begin{minted}[linenos=true,breaklines=true]{Cypher}
//NP VERB PREP NP NP

MATCH (verb:Node)
WHERE ANY (uuid IN ["UUIDS"] WHERE uuid = verb.uuid)
MATCH (verb)-[rel_nsubj:REL]->(nsubj:Node)
MATCH (verb)-[rel_nmod:REL]->(nmod:Node)
MATCH (nmod)-[rel_prep:REL]->(prep:Node)
MATCH (nmod)-[rel_dobj:REL]->(dobj:Node)

WHERE rel_nsubj.label = "nsubj"
AND rel_dobj.label = "appos"
AND rel_nmod.label =~ 'nmod.*'
AND rel_prep.label = "case"

RETURN nsubj, verb, prep, nmod, dobj;


\end{minted}
\pagebreak

\section*{points\_api/frame\_queries/NP\_VERB\_PREP\_NP\_PREP\_NP.cql}
\begin{minted}[linenos=true,breaklines=true]{Cypher}
//NP VERB PREP NP PREP NP
//here the NP at the end is really a sentence, the verb is returned

MATCH (verb:Node)
WHERE ANY (uuid IN ["UUIDS"] WHERE uuid = verb.uuid)
MATCH (verb)-[rel_nsubj:REL]->(nsubj:Node)
MATCH (verb)-[rel_nmod:REL]->(nmod:Node)
MATCH (nmod)-[rel_prep:REL]->(prep:Node)
MATCH (verb2:Node)-[rel_advcl:REL]-(verb)
MATCH (verb2)-[rel_prep2:REL]-(prep2:Node)

WHERE rel_nsubj.label = "nsubj"
AND rel_nmod.label =~ 'nmod.*'
AND rel_prep.label = "case"
AND rel_advcl.label =~ 'advcl|ccomp'
AND verb2.part_of_speech =~ 'VB.?'
AND rel_prep2.label =~ 'mark|case'

RETURN nsubj, verb, prep, nmod, prep2, verb2;


\end{minted}
\pagebreak

\section*{points\_api/frame\_queries/PREP\_NP\_VERB\_NP.cql}
\begin{minted}[linenos=true,breaklines=true]{Cypher}
//PREP NP VERB NP

MATCH (verb:Node)
WHERE ANY (uuid IN ["UUIDS"] WHERE uuid = verb.uuid)
MATCH (verb)-[rel_nmod:REL]->(nmod:Node)
MATCH (nmod)-[rel_prep:REL]->(prep:Node)
MATCH (verb)-[rel_dobj:REL]->(dobj:Node)
WHERE rel_nmod.label =~ 'nmod:.*'
AND rel_dobj.label = "dobj"
AND rel_prep.label = "case"
RETURN prep, nmod, verb, dobj;


\end{minted}
\pagebreak

\section*{points\_api/frame\_queries/VERB-universal.cql}
\begin{minted}[linenos=true,breaklines=true]{Cypher}
//VERB
//attempts to match the tree for all verbs

MATCH (verb:Node)
WHERE ANY (uuid IN ["UUIDS"] WHERE uuid = verb.uuid)
OPTIONAL MATCH (verb)-[nsubj_rel:REL]->(nsubj:Node) WHERE nsubj_rel.label = "nsubj"
OPTIONAL MATCH (verb)-[nsubjpass_rel:REL]->(nsubjpass:Node) WHERE nsubjpass_rel.label = "nsubjpass"
OPTIONAL MATCH p=(verb)<--(dobjcop:Node)-->(nsubjcop:Node) WHERE ALL (r IN relationships(p) WHERE r.label = "nsubj" OR r.label = "cop")
OPTIONAL MATCH (verb)-[xcomp_rel:REL]->(xcomp:Node) WHERE xcomp_rel.label = "xcomp"
OPTIONAL MATCH (verb)-[dobj_rel:REL]->(dobj:Node) WHERE ANY (l IN ['dobj', 'nmod.*'] WHERE dobj_rel.label =~ l)
RETURN nsubj, nsubjpass, nsubjcop, verb, dobj, dobjcop, xcomp;


\end{minted}
\pagebreak

\section*{utilities/plotter/graph.html}
\begin{minted}[linenos=true,breaklines=true]{HTML}
<!DOCTYPE>
<html>
  <head>
    <title>Points Extraction</title>
    <meta name="viewport" content="width=device-width, user-scalable=no, initial-scale=1, maximum-scale=1">
    <script src="http://code.jquery.com/jquery-2.0.3.min.js"></script>
    <script src="http://cytoscape.github.io/cytoscape.js/api/cytoscape.js-latest/cytoscape.min.js"></script>
    <script src="https://rawgit.com/cytoscape/cytoscape.js-cose-bilkent/master/cytoscape-cose-bilkent.js"></script>

    <style>
      #cy {
        width: 100%;
        height: 100%;
        position: absolute;
        left: 0;
        top: 0;
        z-index: 999;
      }
    </style>

    <script>
      $(function(){
        var cy = window.cy = cytoscape({
          container: document.getElementById('cy'),
          boxSelectionEnabled: false,
          autounselectify: true,
          layout: {
            name: 'cose-bilkent',
            edgeElasticity: 10,
            idealEdgeLength: 150,
            nodeRepulsion: 1000,
            animate: false,
          },
          style: [
            {
              selector: 'node',
              style: {
                'content': 'data(value)',
                'width': 'data(weight)',
                'height': 'data(weight)',
                'text-valign': 'center',
                'text-halign': 'center',
                'color' : 'black',
                'font-size': '25px',
                'background-color': 'YellowGreen',
                'border-width': 2,
                'border-color': 'SeaGreen'
              },
            },
            {
              selector: 'edge',
              style: {
                'label': 'data(label)',
                'width': 4,
                'font-size': '20px',
                'line-color': '#f2f2f2',
                'target-arrow-color': '#ccc',
                'target-arrow-shape': 'triangle',
                'edge-text-rotation': 'autorotate',
              }
            },
            {
              selector: '[type="verb"]',
              style: {
                'shape': 'diamond',
                'background-color': '#ff884d',
                'border-width': 0,
              }
            }
          ],
          "elements":{"nodes":[], "edges":[]}
        });
      });
    </script>
  </head>
  <body>
    <div id="cy"></div>
  </body>
</html>


\end{minted}
\pagebreak

\section*{utilities/plotter/plot.rb}
\begin{minted}[linenos=true,breaklines=true]{Ruby}
# plot.rb
#
# This script takes a list of point patterns and uses the graph HTML template
# to build a resulting HTML file.

require "pry"
require "json"

def sorted_dup_hash(array)
  Hash[*array.inject(Hash.new(0)) { |h,e| h[e] += 1; h }.
  select { |k,v| v > 1 }.inject({}) { |r, e| r[e.first] = e.last; r }.
  sort_by {|_,v| v}.reverse.flatten]
end

data = File.open("summary.txt").readlines.select { |l| l.match(/^[0-9]+ : \[/) }.take(30)
data.map! do |l|
  l = l.downcase.split(" : ")
  { count: l.first.to_i, pattern: l.last.gsub(/\[|\]/, "").chomp }
end

patterns = data.map { |x| x[:pattern].split(" ").map{|x|x.split(".").first }}

words, verbs = [], []
data.each do |point|
  point[:count].times do
    point[:pattern].split(" ").map{ |x| x.split(".") }.each do |word, rel|
      words << word
      verbs << word if rel == "verb"
    end
  end
end
verbs.uniq!

words = Hash[*sorted_dup_hash(words).map { |k, v| [k, Math::log(v).round(2) * 50] }.flatten]
nodes = words.map do |k, v|
  {
    id: k,
    value: k,
    weight: v,
    type: verbs.include?(k) ? "verb" : "",
  }
end

edges = []
patterns.each do |pattern|
  if pattern.size == 2
    edges << { source: pattern.first, target: pattern.last, label: ""}
  elsif pattern.size == 3
    edges << { source: pattern.first, target: pattern.last, label: pattern[1].split(".").first}
  elsif pattern.size == 4
    edges << { source: pattern.first, target: pattern[1], label: ""}
    edges << { source: pattern[1], target: pattern[3], label: pattern[2].split(".").first}
  elsif pattern.size == 5
    edges << { source: pattern.first, target: pattern[2], label: pattern[1].split(".").first}
    edges << { source: pattern[2], target: pattern[4], label: pattern[3].split(".").first}
  end
end
edges.uniq!

in_relationship = []
edges.each do |e|
  in_relationship += [e[:source], e[:target]]
end

nodes.select! { |n| in_relationship.include?(n[:id]) }

nodes.map! { |e| { data: e } }
edges.map! { |e| { data: e } }
elements = Hash[:elements, { nodes: nodes, edges: edges }].to_json[1..-2]

template = File.open("graph.html").read
File.open("output.html", "w").write(template.gsub('"elements":{"nodes":[], "edges":[]}', elements))


\end{minted}
\pagebreak


\end{document}
